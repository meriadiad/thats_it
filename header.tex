% Dokumentenklasse
\documentclass[a4paper, 12pt, twoside, openany]{scrbook}
%
% Neue commands
\newcommand{\semester}{Wintersemester 25/26} % WiSe 2016/17 etc.
% für die Studiengangsbeschreibung:
\newcommand{\Modul}[1]{\texttt{\small #1}}
\newcommand{\VL}[1]{\glqq \textsl{#1} \grqq }
\newcommand{\fach}[1]{\textbf{#1}}
% für Tabellen mit Bezeichnungen:
%\newcommand{\feld}[2]{\textbf{#1}: & #2 \\} % wird nur noch beim Mathetext im Wintersemester gebraucht
% für die Kennzeichnung von Beiträgen
\newcommand{\von}[1]{\begin{flushright}\small{#1}\end{flushright}}
% Ordnernamen als commands zwecks einfacher Änderung
\newcommand{\chapdir}{kapitel}
\newcommand{\imgdir}{bilder}
\newcommand{\NFdir}{\chapdir/nebenfaecher}

% Packages
% Zeichensatz
\usepackage[utf8]{inputenc}
% Sonderzeichen
\usepackage[T1]{fontenc}
% Schriftart
\usepackage{lmodern}
% Deutsch
\usepackage[ngerman]{babel}
% An die Sprache angepasste Anführungszeichen
\usepackage{csquotes}
% Mathezeug
\usepackage{amsmath}
% Mathezeug
\usepackage{amsfonts}
\usepackage{amssymb}
% Bilder
\usepackage{graphicx}
% Ränder
\usepackage[left=2.5cm,right=2.5cm,top=3cm,bottom=4cm]{geometry}
% Kopf- u. Fußzeilen
\usepackage[headsepline]{scrlayer-scrpage}
% pdf einbinden
\usepackage{pdfpages}
% Seiten drehen für Stundenpläne
\usepackage{rotating}
% für Tabellen der Studiengänge
\usepackage{multirow}
% Eurozeichen mit \euro
\usepackage{eurosym}
% für Links
\usepackage[pdfborderstyle={/S/U/W 1}]{hyperref}
% Funktionalität für Abbildungen
\usepackage{caption}

% Layout
% Absatz-Formatierung (keine Einrückung, keine Abstände)
% Einrückung nach Absatz
\setlength{\parindent}{0em}
% vertikaler Abstand nach Absatz
\setlength{\parskip}{0em}

% Chapters und Sections
% die Nummerierungsebene wird um zwei runtergesetzt, nur Subsections bekommen Nummern (ich verwende das hier statt \section*{•} da die * Variante nicht im Inhaltsverzeichnis auftaucht)
\setcounter{secnumdepth}{-2}
% die "Tiefe des Inhaltsverzeichnisses", nur Sections werden angezeigt
\setcounter{tocdepth}{1}
% definiert den Standard Komascript-Befehl um, damit die Überschriften zentriert sind, statt linksbündig
\renewcommand*{\raggedsection}{\centering}

% Kopf- und Fußzeileneinstellungen
% section-Titel in Kopfzeilen: [Text für gerade Seiten] {Text für ungerade Seite}
\automark[section]{section}
% Kopfzeilen Stil
\pagestyle{scrheadings}
% Löschen des Standardinhaltes
\clearpairofpagestyles
% innerer Kopfzeilenslot auf geraden Seiten
\rehead{Fachschaft Physik}
% innerer Kopfzeilenslot auf ungeraden Seiten, hier wird das aktuelle Semester (oben mit dem newcommand-befehl festgelegt) reingeschrieben
\lohead{\semester}
\lehead{\includegraphics[width=0.03\textwidth]{bilder/prof_g.png} \hspace{0.2cm}\headmark}% Äußerer Kopfzeilenslot mit Sectiontitel und Bild auf geraden Seiten
\rohead{\headmark \hspace{0.2cm}  \includegraphics[width=0.03\textwidth]{bilder/prof_u.png}}% Äußerer Kopfzeilenslot mit Sectiontitel und Bild auf ungeraden Seiten
\cfoot{Der Ersti-Leitfaden}% mittlerer Fußzeilenslot
\ofoot{\pagemark}% äußerer Fußzeilenslot, mit Standardbefehl für Seitenzahlen