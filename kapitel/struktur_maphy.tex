\subsection{Master in Physik}
\label{subsec:studienstruktur_msc_physik}
Herzlichen Glückwunsch zur Entscheidung den Physikmaster an der Goethe-Uni Frankfurt zu absolvieren! Hier habt Ihr alle Möglichkeiten, das Masterstudium nach Euren Wünschen zu gestalten.\\
Falls Ihr von einer anderen Hochschule nach Frankfurt gewechselt seid, müsst Ihr eventuell noch vom Prüfungsausschuss auferlegte Veranstaltungen des Bachelors absolvieren. Dies empfiehlt sich natürlich im \textbf{ersten Semester} zu machen. Auf jeden Fall sind aber die \fach{Wahlpflichtmodule}  und das \fach{Nebenfach} für den Einstieg vorgesehen. Über Wahlpflichtmodule müssen 26--30 CP und über Nebenfachmodule 12--16 CP eingebracht werden. Hierbei ist auf jeden Fall eine Summe aus beidem von mindestens 42 CP zu erreichen. Dies ist jedoch bei dem reichhaltigen Angebot weiter schwer. Bei den Wahlpflichtfächern kann man zwischen Veranstaltungen der fünf Institute des Fachbereiches (Institut für Theoretische Physik, Physikalisches Institut, Institut für Angewandte Physik, Institut für Kernphysik und Institut für Biophysik) frei wählen. Bei den Nebenfächern gilt, wie im Bachelor Physik, dass prinzipiell jedes nichtphysikalische Fach der Uni zugelassen ist. Falls das Nebenfach eurer Wahl vorher noch nicht von einem anderen Studierenden belegt wurde, könnt Ihr es beim Prüfungsamt genehmigen lassen, was in der Regel kein Problem sein sollte.\\
Au\ss erdem müsst Ihr ein sogenanntes \fach{Forschungs- und Laborpraktikum} absolvieren. Dies ist eigentlich für das erste Mastersemester vorgesehen. Weil die Anmeldefrist allerdings schon abläuft, bevor Ihr an der Uni startet (normalerweise bis 12.10.), macht ihr das Praktikum einfach nächstes Semester. Bis dahin habt ihr auch einen Praktikumspartner gefunden. Ihr könnt für das Praktikum 2 aus 4 Instituten wählen, wobei in der Semesterhälfte gewechselt wird. Jedoch benötigt ihr für das Biophysikinstitut die Vorlesung \VL{Einführung in die Biophysik}. Solltet Ihr euch also hierfür entscheiden, macht Ihr am besten die Vorlesung im ersten Mastersemester. Des Weiteren müsst Ihr in einem von Euch gewählten \VL{Proseminar} einen Vortrag halten.\\
Nach erfolgreicher Absolvierung einiger Wahlpflichtmodule sollte eure Präferenz hinsichtlich eines Masterarbeitsthemas ein bisschen klarer geworden sein. Somit könnt Ihr euch ab dem 2. bzw. 3. Semester auf die Suche nach einer Arbeitsgruppe für eure Masterarbeit machen. Unterstützend werden auch regelmä\ss ig im Semester die Arbeitsgruppen der Institute vorgestellt, Informationen dazu findet ihr auf der Fachschafts-Website. Das 3. und 4. Mastersemester stehen dann ganz eurer Masterarbeit zur Verfügung, zu der in der Regel auch die Module \fach{Fachliche Spezialisierung}, \fach{Erarbeiten eines Projekts} und das \VL{Arbeitsgruppenseminar} gehören.