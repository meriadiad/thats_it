\section{Master in Biophysik}
\bigskip
%
Der Biophysikmaster steht neben Absolventen des Biophysik-Bachelors auch Studierenden mit Abschluss in Physik, Chemie, Biologie oder in einem ähnlichen Studiengang offen. Wobei die Zulassung mit Auflagen erteilt werden kann. Im Gegensatz zum Bachelor gibt es im Master sehr viele Wahlmöglichkeiten.
%
\subsection{Erstes und zweites Semester}
Die ersten beiden Semester sind den Wahl-Pflicht-Vorlesungen vorbehalten. Der 40 CP große Wahl-Pflicht-Bereich ist in drei Schwerpunkte aufgeteilt: Theorie, Methoden und Systeme, wobei aus jedem Module gewählt werden müssen. 20 CP müssen benotet eingebracht werden und aus jedem Schwerpunkt müssen benotete CP eingebracht werden. Die aktuelle Liste der zugelassenen Wahl-Pflicht-Veranstaltungen ist am Prüfungsamt ausgehängt und kann durch Antrag beim Prüfungsausschuss erweitert werden.\\
\bigskip\\
Außerdem sieht der Studienverlaufsplan ein \VL{Forschungs- und Laborpraktikum} für 12 CP vor. Dieses kann in den Fachbereichen 13, 14 und 15 ohne Antrag und in anderen Fachbereichen oder außerhalb der Universität nach Genehmigung durch den Prüfungsausschuss gemacht werden. Die Modulabschlussprüfung hierfür ist ein benotete Praktikumsbericht. 
%
%
\subsection{Drittes und viertes Semester}
In diesen Semestern soll die einjährige Masterarbeit gemacht werden. Diese teilt sich auf die drei Module \VL{fachliche Spezialisierung}, VL{Erarbeiten eines Projektes} und \VL{Masterarbeit} auf, welche in Summe auf 60 CP kommen.\\
\bigskip\\
Zusätzlich sollen zwei Arbeitsgruppenseminare für je 4 CP besucht werden. Eines davon kann durch ein Proseminar zu aktuellen Themen der Biophysik ersetzt werden. 
%
%
\subsection{Prüfungsmodalitäten und Endnote}
Wie für den Bachelor gehören auch im Master die verschiedenen Vorlesungen zu unterschiedlichen Fachbereichen und die Anzahl von Prüfungswiederholungen sowie die Fristen zur Prüfungsanmeldung sind durch die Bestimmungen der jeweiligen Fachbereiche geregelt. Die Zuordnungen der Vorlesungen zu den Fachbereichen sollte in den Modulbeschreibungen zu finden sein.
\bigskip\\
%
Die Endnote für den Master berechnet sich zu 50~\% aus den benoteten Pflicht- und Wahlpflichtmodulen und zu 50~\% aus der Masterarbeit.
%
%


