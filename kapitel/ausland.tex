\section{Studieren im Ausland}
\label{sec:ausland}
Wie viele andere Studierende träumt ihr vielleicht auch davon, ein oder zwei Semester, vielleicht auch etwas länger, im Ausland zu verbringen. Um euch einen Überblick über die verschiedenen Möglichkeiten und die Probleme, die man bei der Planung eines Auslandsaufenthalts hat, zu schaffen, haben wir für Euch ein paar Informationen gesammelt und hier zusammengestellt.

\subsection{Beste Zeitpunkte}
\begin{enumerate}
	\item 3./4./5./6. Semester: möglich aber schwierig.
	\item Die Bachelorarbeit im Ausland anfertigen -- dies machen nicht alle Professoren mit, bei Interesse müsst ihr Euern Jeweiligen fragen
	\item 6./7./8. Semester: mit dem Beginn des Masterstudiums -- am entspanntesten.
	\item Die Masterarbeit im Ausland anfertigen -- auch hier solltet ihr euern Betreuer/Professor ansprechen.
\end{enumerate}

\subsection{Vor- und Nachteile}
\begin{enumerate}
	\item Im Bachelorstudium wird es schwierig(er) sein, sich die im Ausland erbrachten Leistungen anerkennen zu lassen, da durch das Bachelorstudium sehr genau festgelegt ist, welche Vorlesung mit welchem Inhalt wieviele Creditpoints erbringen soll. Wer darauf verzichten kann, in der Regelstudienzeit fertig zu werden, kann es aber versuchen und kann im schlimmsten Fall ein Studiensemester oder -jahr \enquote{verlieren}. Dafür lernst du aber schon sehr früh im Studium viel darüber, wie man sich auch in ungewohnten Umgebungen orientiert und wie es \enquote{woanders läuft}.
	\item Die Bachelorarbeit im Ausland -- sofern möglich -- ist leider meistens relativ kurz ($\approx$ 3-5 Monate). Außerdem hat man sehr genaue Vorgaben, was man machen muss. Um das (Wunsch-)Land selbst kennen zu lernen, bleibt meist nicht so viel Zeit.
	\item Der Masterstudiengang bietet durch seine Flexibilität (im Modul \enquote{Wahlpflicht} kann man viel flexibler Vorlesungen einbringen) eher Möglichkeiten, Studienleistungen aus dem Ausland anerkennen zu lassen. Außerdem gibt es keinerlei Pflichtvorlesungen, an denen man die Zeit im Ausland ausrichten müsste.
	\item Die Masterarbeit im Ausland -- auch hier gilt, wie bei der Bachelorarbeit, dass evtl. wenig Zeit bleibt, das neue Land kennen zu lernen. Dafür ist die Masterarbeit -- in der Regel $\frac{1}{2}$ Jahr Einarbeitung + $\frac{1}{2}$ Jahr schreiben = 1 Jahr -- länger und bietet dadurch eher mehr Zeit.
\end{enumerate}

\subsection{Woher Informationen?}
Für eigentlich alle Angelegenheiten ist das \enquote{International Office} im Westend der geeignete Ansprechpartner. Informationen findest du unter \url{http://www.uni-frankfurt.de/international/} oder in der Sprechstunde des International Office. Eine Übersicht über die Sprechstunden findest Du auf der Internetseite.

\subsection{Wie geht das eigentlich?}
Generell gibt es drei Möglichkeiten:
\begin{enumerate}
	\item Erasmus: Alle Länder der Europäischen Union sowie Beitrittskandidaten und einige weitere Länder (bspw. Schweiz, Norwegen) nehmen am Erasmus-Programm teil. Welche Universitäten für dich in Frage kommen, findest du auf der Homepage des International Office.
	\item nicht-Erasmus: Für fast alle anderen Länder gibt es Förderprogramme des DAAD (Deutscher Akademischer AustauschDienst) -- diese Förderprogramme sind in der Regel höher dotiert als Erasmus, aber auch etwas schwieriger zu bekommen.
	\item Alles auf eingene Faust: Du kannst natürlich einfach alles selbst machen -- vergibst damit aber wahrscheinlich die Chance auf ein Auslandsstipendium nach Erasmus oder DAAD (Ausnahme: evtl. \enquote{Auslands-Bafög}). Dafür kannst du im Prinzip, wohin du willst.
\end{enumerate}

\subsection{Wann sollte man sich Gedanken machen?}
Grundsätzlich sollte man sich etwa anderthalb Jahre vor dem geplanten Auslandsaufenthalt schon mal erkundigen. Etwa ein Jahr vorher kann man sich dann anmelden, hat dann auch für alles noch genug Zeit. Generell kann man nichts zu früh machen! Wer zuerst kommt, bekommt die Plätze -- wer also spät oder spontan noch \enquote{mal eben schnell} ins Ausland will, hat entweder keine große Auswahl oder gar keine Chance mehr auf einen Platz. Im Zweifel gilt aber immer: Mutige vor! Es soll auch Leute geben, die ihr Auslandssemester in ein paar Wochen organisiert haben\ldots erzählt es nur nicht den Leuten im International Office ;-)

\subsection{Warum überhaupt?}
Da ihr ja eben erst an der Uni angekommen seid, macht es wohl mehr Sinn, über die dritte Variante, einen Auslandsaufenthalt nach dem Bachelor, nachzudenken. Dabei sollte man sich darüber klar werden, warum man das machen möchte. Denn über das \enquote{Warum} wird auch die Wahl des \enquote{Wos} leichter. Eins ist jedoch klar: Es wird euer Studium nicht beschleunigen, bestenfalls ist man genau so schnell. Dem Verfasser dieser Zeilen ist jedoch kein Student und keine Studentin bekannt, die ihr Auslandssemester oder -jahr bereut hat -- ganz im Gegenteil. In der Physik gibt es zwar keinen besonderen Grund, aus Lehrgründen in bestimmte Länder zu gehen, weil dort \enquote{eine andere Art Physik} gelehrt wird -- wie das vielleicht bei Philosophie oder Politik möglich sein kann. Der eigentliche Grund wird daher eher \enquote{Erweiterung des geistigen Horizonts} oder -- simpler ausgedrückt -- Spaß sein. Wahrscheinlich wird man nie wieder in seinem Leben so viel Zeit haben, soviele interessante Leute aus der ganzen Welt kennenlernen und gemeinsam ein fremdes Land entdecken zu können. Ob man nun ins kalte Wasser (Irlands, Norwegens, Polens oder Estlands) springt, die Sprache nicht kann und versucht, gerade so irgendwelche Vorlesungen zu verstehen -- oder ins warme Wasser (Südspaniens, Italiens, Kroatiens, Griechenlands oder Bulgariens) springt, schon aus Schulzeiten fließend die Landessprache spricht und schneller Kontakte knüpfen kann: es wird sehr wahrscheinlich eine großartige Zeit sein, in der ihr viel Spaß haben könnt. Sie wird sehr wahrscheinlich das Bild des Landes, in dem ihr studiert, verändern; man lernt unheimlich viele Dinge, die keine Universität lehrt -- und man bekommt auch einen anderen Blick auf \enquote{Zuhause}. Ganz abgesehen davon macht sich das ganze natürlich im Lebenslauf super...

\subsection{Was kostet es?}
\paragraph{USA} An den Top-Unis der USA muss man mit bis zu 22.000,-- \euro{} (3 Nullen!) für ein akademisches Jahr (ca. 9 Monate) rechnen. Es gibt allerdings kleinere Unis, die sich nach Studierenden aus dem Ausland förmlich reißen, um für einen besseren Ruf zu sorgen. Sie verlangen teilweise keine Tests und Studiengebühren. Fragt sich halt, ob man dort hin möchte. An den "`normalen"' Unis kann man sich aber absolut nicht darauf verlassen, finanziell in irgendeiner Art unterstützt zu werden. Es ist ratsam, einige Unis (z.B. per E-Mail) anzuschreiben und um Informationsmaterial für die Graduate-Studiengänge zu bitten. Das Amerikahaus veranstaltet übrigens gelegentlich Informationsabende zu diesem Thema.

\paragraph{Europa} In Europa dagegen sieht die ganze Sache schon besser aus: Wenn man mit Erasmus auf Entdeckungsreise geht, werden einem die evtl. anfallenden Studiengebühren erlassen. Dazu gibt es ein kleines Stipendium (2011: 150 \euro{}/Monat). Die Höhe des Stipendiums hängt jedoch empfindlich von der Anzahl der AuslandsstudentInnen ab. Darüber hinaus braucht man dann entsprechend das Geld, das im jeweiligen Land zum Leben reichen muss. Skandinavien und die Schweiz sind da tendenziell sehr viel teurer als Deutschland, Osteuropa eher etwas billiger und daher auch vor Ort \enquote{leicher zu entdecken}.

\paragraph{Rest der Welt} Es soll ja noch mehr als nur die USA und Europa geben: im Rest der Welt gibt es natürlich auch Unis! Dort ist der ganze Austauschprozess jedoch normalerweise nicht so sehr institutionalisiert wie in Europa, es gibt bspw. kein Erasmus. Der DAAD (siehe auch oben) hilft einem da aber finanziell und organisatorisch. Die Kosten hängen dort sehr stark davon ab, ob man die Studiengebühren \enquote{weg-verhandeln} kann und wie hoch die Lebenshaltungskosten des Landes sind. Auf der Homepage des DAAD gibt es eine Liste zu Fördersätzen und Reisekostenzuschüssen. Hier ist beinahe jedes Land, das mindestens eine Universität hat, möglich.

\subsection{Leistungen}
Auch hier sollte man sich sicherheitshalber vorher gründlich informieren. Nicht alle Leistungsnachweise einer Uni im Ausland werden zu Hause anerkannt, eventuell merkt man dann erst später, dass man das ein oder andere hier noch einmal machen muss. Mit den neuen internationalen Bachelor- und Master-Studiengängen ist es auch innerhalb Europas nicht einfacher geworden. Sprecht vorher mit dem entsprechenden Koordinator oder der Koordinatorin. Dazu sucht ihr euch am besten schon einmal die Liste alle möglichen Vorlesungen heraus.

\subsection{Sprachtests}
\paragraph{USA und England}
In den USA und in England ist es die Regel, dass man bei der Anmeldung sein TOEFL-Ergebnis vorlegen muss. TOEFL (Test Of English as a Foreign Language) ist ein Sprachtest, bei dem man Punkte bekommt. Je besser, desto mehr Punkte. Mit etwas Vorbereitung ist er allerdings gut zu schaffen. An diesen Test sollte man früh genug denken, da es vom Abschicken der Anmeldung bis zum Test und dann noch mal bis zur Punktebenachrichtigung 4-5 Monate dauern kann! Das Ergebnis ist 2 Jahre lang gültig.

\paragraph{Rest der Welt}
Das International Office und auch der DAAD verlangen bei \enquote{verbreiteten Sprachen} -- Englisch, Spanisch, Französisch, evtl. mehr -- eine Bewerbung und ein Motivationsschreiben in Landessprache sowie ein irgendwie gearteter Nachweis über Können der Landessprache. Bei kleineren Ländern und weniger verbreiteten Sprachen -- z.B. Polnisch, Tschechisch, Bulgarisch, Portugiesisch, ... -- wird ein Motivationsschreiben in Englisch sowie ein kleiner -- an der Uni absolvierbarer -- Englisch-Sprachtest erwartet.

\subsection{Hinweis zum Schluss}
Da sich vieles mit den Austauschen im stetigen Wandel befindet, können die Informationen hier falsch, veraltet oder irreführend sein -- aktuell und hilfreich ist immer das International Office. Schaut einfach unter \url{http://www.uni-frankfurt.de/international/} nach, wann und wo sie ihre Sprechstunde haben -- und löchert sie dort mit Fragen. Außerdem gibt es nicht so endlos viele Physikstudierende, die ins Ausland gehen, aber viele Programme. Traut Euch und fragt -- Nur Mut! :-)
\von{Fips}
\newpage
\begin{figure}[!b]
	\centering
	\includegraphics[height=\textheight]{\imgdir/cern.jpg}
\end{figure}