\section{Mentorenprogramm}
\textbf{Liebe StudienanfängerInnen,}
\medskip
herzlich willkommen am Fachbereich Physik!

Von der Schule kommend, ist vieles neu für Sie, manches an der Universität erscheint Ihnen vielleicht zunächst schwer zu durchschauen. Als eine Einstiegshilfe, die Ihnen über unnötige
Klippen und Probleme am Anfang hinweghelfen soll, hat der Fachbereich seit einigen Jahren das bewährte Mentorenprogramm eingeführt; Ihnen wird ein/e Hochschullehrer/in zugeteilt,
der oder die Ihr persönliche/r Mentor/in ist. Er ist Ihr direkter Ansprechpartner, der Ihnen für die ersten Semester zur Seite stehen soll. In einem persönlichen Gespräch lassen sich viele Schwierigkeiten des Anfangs oft viel besser als in Informationsveranstaltungen lösen.

Unser Fachbereich, Studierende und Lehrende, ist sehr familiär. Ihr Mentor ist eine Ihrer Kontaktstellen, um die Aufnahme in diesen Kreis zu erleichtern. Ihr Mentor soll nicht nur bei organisatorischen Problemen helfen, sondern Ihnen auch bei anderen möglichen Problemen mit dem Studium beistehen. Warum haben Sie sich für Ihr Physikstudium entschieden? Sprechen Sie mit Ihrem Mentor darüber, er kann Ihnen helfen, Ihre eigenen Ziele sicherer zu erreichen. Auch wenn er Ihnen nicht direkt helfen kann, wird er oft doch Kontakte vermitteln können. Sie haben Probleme mit der Finanzierung des Studiums? Vielleicht kann Ihnen Ihr Mentor eine Stiftung oder ein Stipendium empfehlen. Sie wollen ins Ausland? Ihr Mentor kann Ihnen viele Tipps geben. Sie fühlen sich vom Studium überfordert und wollen schon nach 2 Monaten alles
hinschmeißen? Sprechen Sie mit Ihrem Mentor darüber, er kann Ihnen helfen, selbstkritisch zu prüfen, ob Physik das Richtige für Sie ist. Sie haben Ideen, was man am Studium verbessern
könnte? Wo klemmt es am meisten? Ihr Mentor wird helfen, dass Ihre Anregungen aufgenommen werden.

Die Qualität des Studienangebotes lebt auch von Ihrem kritischen Feedback. Hier hat auch Ihr Studiendekan (zur Zeit der Autor dieses Textes) immer ein offenes Ohr.

Da der Fachbereich aus Gründen des Datenschutzes Ihre Kontaktdaten nicht hat, bitten wir Sie, sich bei der Einführungsveranstaltung auf die dort ausgelegte Liste für das Mentorenprogramm einzutragen. Geben Sie Ihre Adresse, Telefonnumer und E-Mail an. Ihre Adressen werden dann
zufällig auf die Hochschullehrer verteilt. In den kommenden 3 Wochen sollten Sie dann eine Einladung zu einem Gespräch bekommen. Nehmen Sie die Gelegenheit wahr, lernen Sie gleich
einen der Hochschullehrer etwas näher kennen und stellen Sie alle Fragen, die auftauchen, diskutieren Sie die Probleme, die Sie am Anfang haben.

Falls Sie innerhalb von 3 Wochen keine Einladung erhalten haben oder es verpasst haben, sich auf die Liste einzutragen, melden Sie sich bitte beim Dekanat des Fachbereiches Physik (798 47205, dekanat@physik.uni-frankfurt.de)

Einen guten Start in Ihr Physikstudium wünscht:

\von{
Prof. Dr. Joachim Maruhn\\
ehemaliger Studiendekan Fachbereich Physik
}
