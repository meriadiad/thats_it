\section{Studienfinanzierung}
\subsection{BAföG}
Das Bundesausbildungsförderungsgesetz (BAföG) ist den meisten wahrscheinlich bekannt. Es ist eine staatlich finanzierte Beihilfe für Studierende, deren Höhe abhängig von eurem und dem Einkommen eurer Eltern berechnet wird. Der Höchstsatz liegt bei knapp 600~\euro{} pro Monat, zusätzlich gibt es Zuschüsse zur Krankenversicherung, Auslandssemester u. Ä.. Ein Antrag lohnt sich auf jeden Fall, denn verlieren könnt ihr nichts - im schlimmsten Fall bekommt ihr bescheinigt, dass euch aufgrund eurer finanziellen Situation keine staatliche Beihilfe zusteht. Informationen zum Antrag findet ihr auf der Webseite \url{www.studentenwerkfrankfurt.de}.

\subsection{Stipendien}
Eine gute Alternative zu BAföG, die viel weniger bekannt ist, sind Stipendien. In Deutschland gibt es 13 staatlich finanzierte Förderwerke und zahlreiche private Stiftungen, die Studierende finanziell und ideell fördern: Die Höhe der Geldförderung bei staatlichen Förderwerken unterliegt auch dem Bundesausbildungsförderungsgesetz, zusätzlich bekommt ihr aber unabhängig vom Elterneinkommen 300~\euro{} Büchergeld. Unter ideeller Förderung versteht man Seminare und Sommerschulen, die die Förderwerke für ihre Stipendiaten organisieren. Was ihr im Gegenzug erbringen müsst, sind gute Leistungen im Studium (Achtung, lasst euch nicht abschrecken – das hei\ss t nicht, dass unbedingt eine 1 vor dem Komma stehen muss!!) und soziales/politisches/gesellschaftliches Engagement. Dazu zählen zum Beispiel aktive Mitarbeit in der Schülervertretung oder einem gemeinnützigen Verein, wobei auch die persönlichen Umstände berücksichtigt werden. Ein weiterer Vorteil: Das Stipendium müsst ihr am Ende des Studiums nicht zurückzahlen. Ausführliche Informationen über die 13 staatlichen Förderwerke findet ihr unter \url{www.stipendiumplus.de}.\\
Neben diesen gro\ss en Förderwerken gibt es noch private Stiftungen, die Stipendien für verschiedenste Zwecke vergeben. Hier sind die Anforderungen sehr unterschiedlich. Zum Beispiel gibt es Stipendien für soziale Projekte oder für Abschlussarbeiten im bestimmten Bereich. Informationen hierzu findet ihr auf \url{www.mystipendium.de}.\\
Die Hürden sind in Wirklichkeit nicht so gro\ss  und die Stiftungen schöpfen ihre zur Verfügung stehenden Mittel oft gar nicht aus, deshalb lohnt sich ein Versuch!

\subsection{Studienkredit}
Alternativ gibt es noch die Möglichkeit, einen Studienkredit aufzunehmen. Zum Beispiel bietet die KfW Studienkredite zu günstigen Konditionen an. Für weitere Informationen schaut unter \url{www.studienkredite.org}.