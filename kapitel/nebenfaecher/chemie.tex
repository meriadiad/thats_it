\subsection{Chemie}
\label{subsec:chemie}
Das klassische Physiker-Nebenfach ist eindeutig \fach{Chemie}. Das ist nicht verwunderlich, da sich die Themen oft gut erg\"anzen. \fach{Chemie} geh\"ort au\ss erdem zu den Nebenf\"achern mit den meisten Modulen, sodass man sich aus einer grossen Auswahl zusammenstellen kann, was einen am meisten interessiert. Oder man kombiniert mit einem anderen Nebenfach. Auf jeden Fall muss man die Vorlesung \VL{Allgeime und anorganische Chemie} (Modul \Modul{ChemA}) sowie ein Praktikum (Modul \Modul{ChemB} oder \Modul{PCP}) besuchen. Alle weiteren Module sind danach optional.\\
Ein grosser Vorteil dieses Nebenfachs ist zweifellos auch, dass die Chemie-Geb\"aude \enquote{nebenan} ~sind, d.h. Campus-Pendeln entf\"allt.\\\\
1.Semester:\\
Vorlesung \VL{Allgeime und anorganische Chemie} (Modul \Modul{ChemA}, 7CP).\\
In dieser Vorlesung geht es allgemein um die Grundlagen der Anorganischen und Physikalischen Chemie. Im Prinzip: Atome und Molek\"ule, Thermodynamik, chemisches Gleichgewicht, S\"auren und Basen, Elektrochemie und nat\"urlich einmal quer durchs Periodensystem - eine Auffrischung der kompletten Schulchemie und mehr.\\
Am Ende des Semesters wird eine Klausur geschrieben, nach deren Ausgang die Praktikumspl\"atze vergeben werden. Diese Klausur ist aber kein Hindernis - in der Regel schneiden Physiker dabei ganz gut ab.\\
%Die Vorlesung selber teilen sich zwei Professoren: Prof. Auner, Prof. Martin Schmidt und Prof Buchsbaum. Gewechselt wird um Weihnachten herum. Der erste Teil ist eher etwas trocken, aber der zweite Abschnitt kompensiert das ganz gut. Dann werden auch mal Experimente vorgef\"uhrt und Anschauungsmaterial mitgebracht.\\
Wer also schon immermal Silizium-Wafer in der Hand haben wollte, oder die Kristalle der verschiedenen Elemente und Verbindungen sehen wollte, der ist hier gut aufgehoben. Einige interessante Themen sind:\\
Gewinnung von Aluminium und Silizium, warum Diamant nicht ewig h\"alt, Funktionsweise von Laserdruckern und Kopierern, Photographie, Halbleiter... um nur ein paar Dinge zu nennen.\\\\
Nachteile will ich auch nicht verschweigen: Zumindest am Anfang ist der H\"orsaal ziemlich voll, da diese Vorlesung f\"ur andere Studieng\"ange Pflichtveranstaltung ist. Ausserdem liegt die Vorlesung f\"ur Langschl\"afer und Weitgereiste ung\"unstig, n\"amlich gleich Montag und Mittwoch von 8-10 Uhr.\\
Zu dieser Veranstaltung wird eine \"Ubung (Teilnahme freiwillig) angeboten.\\\\
2.Semester:\\
Praktikum \VL{Chemie f\"ur Naturwissenschaftler} (\Modul{ChemB}, 3,5CP)\\
Das Praktikum ist vierst\"undig angesetzt, aber wenn man es gut organisiert, braucht man wesentlich weniger Zeit. Es geht dabei haupts\"achlich um anorganische Chemie und ihre Anwendungen, wie z.B. Titration, Nachweisreaktionen, pH-Wert-Bestimmung und Trennmethoden. Insgesamt sind es acht Versuchstage und jeweils ein Seminar dazu. Auch hier wird am Ende wieder eine Klausur geschrieben, deren Stoff aber wesentlich weniger umfangreich ist als im ersten Semester. Bei den Versuchstagen ist Kitteltragen durchaus sinnvoll, sonst hat man pl\"otzlich L\"ocher in der Kleidung, wo eigentlich keine sein sollten. Insgesamt macht es viel Spa\ss , die verschiedenen L\"osungen zusammenzumixen und man erlebt dabei immer wieder \"Uberraschungen.\\
Als Alternative kann das Praktikum \VL{Physikalische Chemie} besucht werden (\Modul{PCP}, 5,5CP).\\\\
Hier noch eine kleine Auswahl weiterf\"uhrender Module:\\
\Modul{Physikalische Chemie I} (Therm, 6CP),\\
\Modul{Organische Chemie} (Org, 7CP),\\
\Modul{Festk\"orperchemie} (ChemF, 3CP).\\\\
Chemie ist auf jeden Fall ein lohnendes und interessantes Nebenfach, wenn man ein bisschen panschen will und sich auch f\"ur andere Naturwissenschaften interessiert, denn Chemie verbindet im Prinzip Physik, Biologie und Geowissenschaften miteinander. Super.
\von{Kristin}