\subsection{Informatik}
%Im Unterschied zur Physik, die beschreibt "`was ist``, versucht die Informatik, Handlungsanweisungen und Konzepte zu geben, um grosse Datenmengen (Informationen) effizient verarbeiten zu können. Dazu werden in der Vorlesung grundlegende Konzepte für Handlungsanweisungen (Algorithmen), ihre sprachliche Formulierung (Programme) und ihre Informationen (Daten) vorgestellt und gelernt. Dies dient dazu, Struktur und Design sowie den Einsatzbereich verschiedener Programmiersprachen zu erkennen und einschätzen zu können, um für ein gegebenes Problem die geeignete programmtechnische Formulierung zu wählen. Aber auch den Lebenszyklus von Software und elementare Prozesse und Methoden der Software--Entwicklung lernen wir kennen. Weiterhin wird die Ablaufumgebung, also die typischen Konzepte und Eigenschaften von Betriebssystemen, erläutert, um bei Problemen konstruktiv eingreifen zu können. Dazu zählen beispielsweise die Problemfelder IT--Sicherheit und Netzwerke. Im zweiten Teil der Vorlesung werden die Programmiersprachenkonzepte von Syntax und Semantik um die Bereiche der funktionalen und objektorientierten Sprachen erweitert und damit das Verständnis von Programmiersprachen vertieft. Dazu kommt die Modellierung, Verwaltung und Nutzung großer Datenbestände mit Hilfe von Datenbanken.
%\von{Rüdiger Brause}


\fach{Informatik} ist ein durchaus nützliches Nebenfach, da das Programmieren oft ein wichtiges Hilfsmittel
für Physiker darstelllt. 
So vielseitig wie der Studiengang Informatik kann auch dieses Nebenfach werden, 
denn es lassen sich alle Module wählen und beliebig kombinieren. 
Will man das Fach einbringen, muss man jedoch die Vorlesung \VL{Grundlagen der Programmierung 1} (\Modul{PRG-1})
mit Übung belegen und die Klausur, die meist noch in der Vorlesungszeit geschrieben wird, bestehen.
Diese Vorlesung gibt einem einen guten und strukturierten Überblick über die Themengebiete der Informatik, 
sie ist aber gleichzeitig auch ein Schnupperkurs zur Programmierung.
Belohnt wird dies mit 11 CP.
Anschließend kann man, zum Beispiel im Sommersemester,
mit \VL{Datenstrukturen} oder \VL{Hardwarearchitekturen und Rechensysteme} weitermachen.
Das Nebenfach ist ab dem ersten Semester geeignet.
Da \Modul{PRG-1} nur im Wintersemester angeboten wird, ist es auch sinnvoll, damit gleich anzufangen.
\fach{Informatik} ist auf jeden Fall ein interessantes Nebenfach, sowohl für diejenigen mit Vorkenntnissen,
um diese zu strukturieren und zu vertiefen, als auch für diejenigen ohne Vorkenntnisse,
um einen Überblick über das Fachgebiet zu erhalten.
\von{Rita}

\begin{center}

  \includegraphics[width=0.35\textwidth]{bilder/einstein.jpg}

\end{center}
