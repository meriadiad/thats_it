\subsection{Geschichte der Naturwissenschaften}

Geschichte der Naturwissenschaften - was soll das den sein?
... Habe ich mir auch gedacht und bin einfach mal in der ersten Woche hingegangen.\\
Dieses Nebenfach besteht aus mehreren kleinen Seminaren mit jeweils zwei Stunden die Woche, ist also ziemlich gut neben den Pflichtvorlesungen zu machen.\\

Nun aber zum Inhalt:
In \fach{Geschichte der Naturwissenschaften} geht man der naturwissenschaftlichten Begriffs- und Theorienbildung auf den Grund.
Woher kommen zum Beispiel die Begriffe \textit{Kraft} und \textit{Energie}?
Wie hat sich unser Weltbild im Laufe der Jahrtausende langsam entwickelt?
Worin sind unsere heutigen Denkstrukturen begründet?
Wenn ihr diese Fragen interessant findet, ist \fach{Geschichte der Naturwissenschaften} wirklich sehr zu empfehlen.\\

Im Moment unterteilt sich das Fach noch in zwei Module, von denen eines 12 CP und eines 13 CP liefert:\\
Das 12 CP - Modul trägt den Titel \Modul{Einführung in die Naturphilosophie}.
Wie der Name schon sagt, beschäftigt man sich hier in drei eher philosophischen Seminaren mit den Anfängen wissenschaftlichen Denkens.
Haupsächlich wird hier die Antike behandelt.\\
Das 13 CP - Modul heißt \Modul{Einführung in die Geschichte der Naturwissen- schaften}.
Auch dieses Modul teilt sich wieder in drei Seminare auf, deren Themen von Semester zu Semester variieren.
Relativ oft wird die klassische Mechanik im 17. und 18. Jahrhundert behandelt.
Ansonsten betrachtet man die Originaltexte einzelner Mathematiker und Physiker dieser Zeit.
Oft werden die Texte auch in der Original-Sprache vorgelegt, interessant, wenn man in der Schule Französich oder Latein gelernt hat.
Das ist allerdings keine Voraussetzung, es gibt auch immer Übersetzungen zumindest ins Englische.\\

Jedes Seminar ist benotet.
Wie die Note zustande kommt, hängt allerdings am Dozenten:
Es kann also ein Kurzvortrag sein oder auch eine Hausarbeit.\\

Das Nebenfach lässt sich in drei Semestern abschließen.
Alles in allem ist \fach{Geschichte der Naturwissenschaften} ein sehr angenehmes Nebenfach, das eine willkommene Abwechslung zum Pflichtbereich ist.

\von{Miriam}
