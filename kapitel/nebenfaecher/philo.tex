\subsection{Philosophie}

\textit{"`Ich weiß, dass ich nichts weiß!"'} \\
Das wissen eigentlich alle oder?
Warum also noch Philosophie studieren?\ldots
Warum eigentlich \"uberhaupt irgendetwas machen?
Warum handeln Menschen oder erst einmal:
Gibt es uns Menschen denn? und wenn ja, was gibt es noch? 
\begin{figure}[!b]
  \begin{center}
    \includegraphics[width=.275\textwidth]{bilder/Ich_weiss.jpg}
  \end{center}
\end{figure}

Viele Fragen gibt es in der Philosophie, das ist wohl wahr\ldots
Aber halb so wild, denn was ist schon "`Wahrheit"'?\\
\\
W\"ahlst du \fach{Philosophie} als Nebenfach, wirst du dich mit all diesen Fragen besch\"aftigen m\"ussen.
Du wirst \"uber die Metaphysik und die Erkenntnistheorie diskutieren
und du wirst dir bez\"uglich so mach einer Meinung die Hand vor den Kopf schlagen oder erstaunt die Augen \"offnen. \\
Das Philosophiestudium ist eines der vielseitigsten und wenn du denken und diskutieren magst,
auch eines der spaßigsten \"uberhaupt.
Und als Erg\"anzung f\"ur die sonst so klinischen Physiker-Nerds ideal geeignet. \\
Das gilt \"ubrigens auch anders herum.
Es ist erstaunlich, wie oft dir in der Philosophie die Physik begegnet.
Mein Tutor zog die Quantenmechanik mindestens einmal pro Stunde als Beispiel heran =)

Das Philosophiestudium beginnst du mit einem der Basismodule (\Modul{BM}).
Im \Modul{BM1} hast du 2x2 SWS Vorlesung und ein 1x2 SWS Tutorium.
Die Vorlesung sieht folgendermaßen aus:\\
Jeder Student kauft sich eine Mappe mit philosophischen Texten (auch online verf\"ugbar), welche gelesen werden m\"ussen und nach und nach in den Vorlesungen besprochen werden.
Im Tutorium sieht man sich den Text dann noch genauer an und merkt, wie wenig man ihn eigentlich verstanden hat :P\\
Abgeschlossen wird das Modul durch eine Klausur und bringt dir dann 12 CP.\\
Auch sehr interessant ist es als Physiker \Modul{Logik} zu h\"oren. 
Hier lernt man n\"amlich, wie man logisch richtige Argumente hervorbringt und was zum Beispiel logische Fehlschl\"usse sind. \\
Dazu kannst du eines der \Modul{AM} (Aufbaumodule) belegen, welche 8CP's bringen.
Welche \Modul{AM's} angeboten werden, kannst du im aktuellen Vorlesungsverzeichnis nachlesen.
Als Besonderheit ist zu beachten, dass du als Nebenf\"achler alle Aufbaumodule besuchen darfst,
auch wenn sie als Vorraussetzung Vorlesungen haben, die du nicht geh\"ort hast.
\von{Marco}
