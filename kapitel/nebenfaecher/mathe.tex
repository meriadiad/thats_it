\section{Mathematik f\"ur Studierende der Physik}

%\begin{minipage}{.60\textwidth}
% \begin{tabular}{l p{0.7\textwidth}}
%   \feld{Was}{Mathematik f\"ur Studierende der Physik 1}
%   \feld{Wer}{Dr. Sven Jarohs}
%   \feld{Wann/Wo}{Di, Do 8-10, OSZ H1}
% \end{tabular}\\
% \end{minipage}
%
% \hfill  \rule{\textwidth}{0.1pt}
% \bigskip
 
\begin{tabular}{p{0.2\linewidth}p{0.79\linewidth}}
  \textbf{Was:} & Mathematik f\"ur Studierende der Physik 1\\
  \textbf{Wer:} & Dr. Sven Jarohs\\
  \textbf{Wann/Wo:} & Di, Do 8-10, OSZ H1\\
\end{tabular}\\
\rule{\textwidth}{0.1pt}
\bigskip 
 
Als Grundlage zur Beschreibung physikalischer Vorg\"ange sind fundierte Kenntnisse in weiten Bereichen der Mathematik unabdingbar. Wesentliche Durchbr\"uche in 
der Physik (z.B. Newtonsche Mechanik) gingen immer einher mit entsprechenden Entwicklungen in der Mathematik (z.B. Leibnizsche Differentialrechnung). In der Vorlesung 
\textsl{Mathematik f\"ur Studierende der Physik 1} werden uns viele aus der Schule bekannte Sachverhalte (z.B. Vektorrechnung, Differentialrechnung, Integration etc.) erneut begegnen,
die allerdings auf deutlich h\"oherem Niveau behandelt werden. Dabei ist es unser Ziel, einen sicheren Umgang mit Grundlagen aus der Analysis und linearen Algebra zu erlernen
 und abstraktes Denken, logisches Schlie\ss en und Beweisf\"uhrung einzu\"uben.
Die Vorlesung \VL{Mathematik f\"ur Studierende der Physik 1} umfasst Themen aus der linearen Algebra und der Analysis, wobei wir im Gegensatz zu den Einf\"uhrungsvorlesungen in der Mathematik
zugunsten einiger physikalischer Beispiele und Anwendungen auf einen vollst\"andig l\"uckenlosen axiomatischen Aufbau und auf den Beweis einiger Aussagen verzichten. \medskip\\
Im Einzelnen werden wir zun\"achst die grundlegende Eigenschaften der Aussagenlogik, der reellen Zahlen, nat\"urlichen Zahlen und rationalen Zahlen behandeln. Vertieft werden wir uns dann mit komplexe Zahlen, Vektorr\"aume (lineare Unabh\"angigkeit, Koordinatenwechsel, Norm und Skalarprodukt),
 Konvergenz, Potenzreihen, Stetigkeit und Differenzierbarkeit vektorwertiger Funktionen,
Integralrechnung und Fourierreihenentwicklung besch\"aftigen.
Dabei werden wir an einigen Stellen bereits einige Grundlagen f\"ur die weiterf\"uhrenden 
Vorlesungen \VL{Mathematik f\"ur Studierende der Physik 2 und 3} bzw. \VL{Mathematik f\"ur Studierende der Meteorologie} legen. \medskip\\
Zu der vierst\"undigen Vorlesung wird ein zweist\"undiges Tutorium angeboten f\"ur das Sie sich am Ende der ersten Woche anmelden k\"onnen.
Vorausetzung f\"ur die Teilnahme an der Abschlussklausur ist die regelm\"a\ss ige und 
erfolgreiche Bearbeitung der \"Ubungsaufgaben (50\% der Punkte und Vorrechnen). Die Abschlussklausur ist benotet, es gehen jedoch nur die beiden besten 
Noten aus den Modulen Mathematik f\"ur Studierende der Physik 1 bis 3 in die Bachelor-Endnote ein. F\"ur die Vorlesung werden wir die Lernplattform
 Physik-Online nutzen. Dort wird im Laufe des Semesters ein Skript hochgeladen und es werden \"Ubungen sowie weitere Informationen zur Verf\"ugung gestellt. Au\ss erdem k\"onnen Sie auf der Lernplattform in einem Forum Fragen zur Vorlesung und den \"Ubungen stellen.
\begin{small}
\textbf{Literaturvorschl\"age:}
\begin{itemize}
\item Beutelsbacher: Lineare Algebra
\item Fischer/Kaul: Mathematik f\"ur Physiker 1
\item K\"onigsberger: Analysis 1 
\item Kreh und Modler: Tutorium Analysis 1 und Lineare Algebra 1
\end{itemize}
\end{small}

\von{Dr. Sven Jarohs, FB 12, Informatik und Mathematik}
\vfill
% \begin{center}
%%  \includegraphics[width=0.8\textwidth]{bilder/youwantproof.jpg}
% \end{center}
 \vfill
