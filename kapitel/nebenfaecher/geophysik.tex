\subsection{Geophysik}
\label{subsec:geophysik}
Als Newton der Apfel auf den Kopf gefallen ist, kam ihm die Idee mit der Schwerkraft. Aber ist die wirklich überall gleich?Und als die Mauern von Troja zusammenstürzten, war das wirklich Poseidon? Und wieso dreht sich die Kompassnadel immer nach Norden? Oder tut sie das gar nicht?\\
All diese Fragen bekommst Du in der Physik, wenn überhaupt, nur am Rande erklärt, weil diese Sachen in den Bereich der \fach{Geophysik} gehören. Die Geophysik beschreibt unseren Planeten mit physikalischen Methoden. Dazu ist die Geophysik in verschiedene Unterbereiche eingeteilt, die sich nur mit einzelnen Teilen beschäftigen, dazu gehören die Bereiche Seismologie, Magnetismus, Schwerebeschleunigung und viele andere.\\
Aber warum soll ich jetzt Geophysik als Nebenfach machen? Und was kann ich damit anfangen? Naja, wenn Dich zum einen unser Planet interessiert, dann bietet es sich an, ein bisschen mehr darüber zu erfahren. Und wenn Du später in der freien Wirtschaft arbeiten willst, hast du mit Geophysik gute Chancen. Mit der Angewandten Geophysik kannst Du zum Beispiel Bodenschätze suchen und finden oder den Untergrund für Hochhäuser erforschen.\\
\begin{figure}[!h]
 	\centering
  	\includegraphics[width=.6\textwidth]{\imgdir/newton.jpg}
\end{figure}
Im ersten Modul Geophysik A (\Modul{GPA}), das Voraussetzung für die anderen beiden ist, erhälst Du mit der Vorlesung \VL{Einführung in die Geophysik} vorab einen Überblick über die Themen. Zu dieser Vorlesung gehört eine Übung, die Ihr besuchen müsst. Dazu kommen noch zwei weitere Vorlesungen, in denen Du Dich in einzelnen Bereichen spezialisierst. Eine der beiden Vorlesungen muss eine Übung haben. Abgeschlossen und benotet wird das Modul mit einer schriftlichen oder mündlichen Prüfung. Für dieses Modul erhälst Du 10 CP.\\
Das zweite Modul Geophysik B (\Modul{GPB}) setzt sich aus drei weiterführenden Vorlesungen zusammen, welche Du Dir ebenso wie die weiteren Vorlesungen im Modul \Modul{GPA} aussuchen kannst. Dadurch erhälst Du einen tieferen Einblick in die einzelnen Bereiche. Zu einer dieser Vorlesungen müsst Ihr an einer Übung teilnehmen. Ein Seminar, in dem Ihr selbst ein Thema auswertet und präsentiert, vervollständigt das Modul. Meistens wird das übergeordnete Thema vorgegeben und Du kannst dann aus einer Reihe von Themen wählen. Dabei betreut Dich einer der Professoren. Abgeschlossen und benotet wird das Modul wieder durch eine mündliche oder schriftliche Prüfung und Du erhälst auch hierfür 10 CP.\\
Als drittes Modul Geophysik C (\Modul{GPC}) gibt es ein Praktikumsmodul, bei dem Du richtige Messungen machst. Dabei kannst Du zwischen einem Laborpraktikum oder einem Feldpraktikum wählen. Bei dem Laborpraktikum machst Du verschiedene Versuche während des Semesters. Zu einem Versuch musst Du ein ausführliches Protokoll schreiben. Dieses Protokoll wird dann kontrolliert und ist der Abschluss dieses Moduls. Entscheidest Du Dich für das Feldpraktikum, fährst über vier Tage ins Gelände, machst Messungen und wertest diese anschlie\ss end aus. Die vier Versuche werden unter Anleitung eines Professors in Gruppen von ca 10 Studierenden gemacht. Zu einem ausführlichen Protokoll, das Abschluss des Moduls ist, musst Du eine weitere Vorlesung aus dem Bereich der Angewandten Geophysik hören. Das Feldpraktikum ist zwar somit mehr Aufwand, lohnt sich aber auf jeden Fall. Für dieses Modul erhälst Du nach dem erfolgreichen Protokoll unbenotet 5 CP.\\
Geophysik ist also die Physik richtig Anwenden und macht eine Menge Spa\ss . Solltest Du Dich für Geophysik als Nebenfach entscheiden, hast Du sicher keine falsche Wahl getroffen.
\von{bur}