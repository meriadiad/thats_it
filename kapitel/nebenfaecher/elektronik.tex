\subsection{Elektronik}
\label{subsec:elektronik}
Das Nebenfach \fach{Elektronik} besteht aus zwei Modulen, die beide belegt werden müssen. Mit beiden kann man innerhalb von zwei Semestern 16 Credit Points bekommen.\\
Das Nebenfach fängt immer im Wintersemester an mit \Modul{Elek1}.
Darin enthalten sind zwei Vorlesungen. \VL{Elektronik und Sensorik} (2 SWS + 1 SWS Übung) im Wintersemester und im darauffolgenden Sommersemester die Vorlesung \VL{Digitalelektronik} (2 SWS). Parallel dazu läuft im Sommersemester ein Praktikum, das Modul \Modul{Elek2}. Das ist wiederum in einen Digital- und einen Analogteil aufgeteilt. Die Credit Points gibt es, wenn man alle Protokolle mit OK zurück hat und eine mündliche Prüfung zu den Vorlesungen hinter sich hat, wobei diese Note als Note für das komplette Nebenfach zählt. In Elektronik ist man meist in einer recht überschaubaren Gruppe, was eine gute Lernatmosphäre schafft.\\
Zu den Themen:\\
Die Analogelektronik beschäftigt sich erst mal mit den grundlegenden elektronischen Bauteilen, wie Widerstand, Spule, Kondensator und den hieraus bestehenden Schaltungen (sogenannte lineare passive Netzwerke). Danach kommen die Halbleiterbauelemente wie Dioden, Bipolartransistoren und Feldeffektransistoren etc dran, sowie eine Einführung in die Sensorik.\\
In der Digitalelektronik werden diese Bauteile dann genutzt um logische Schaltungen aufzubauen, welche mit Boolscher Algebra beschrieben werden können. Au\ss erdem werden Zustandsautomaten, Speicher und Mikroprozessoren besprochen. Hier erhält man sehr interessante Einblicke in das Innenleben von Computern.\\
Allerdings kommt man um einiges an Mathematik nicht drum herum. Deswegen hat man auf jeden Fall einen Vorteil, wenn man schon ein Semester hinter sich hat und zumindest die Grundlagen der Elektrodynamik und Fourier-Analyse kennt. Wen es aber wirklich interessiert, für den sind sicher auch mangelnde Vorkenntnisse kein Problem! Also, einfach mal ausprobieren.
\von{Berit und Martin}