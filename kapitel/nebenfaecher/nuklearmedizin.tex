\subsection{Nuklearmedizin}  %wie viele CP es n\"achstes mal geben wird muss erst noch gekl\"art werden
\label{subsec:nuklearmedizin}
\enquote{Oh mein Gott, wir brauchen Hilfe! Ist ein Arzt anwesend?}\\
Auf diese Frage wolltest du schon immer einmal mit \enquote{Ja!} oder \enquote{Aus dem Weg, ich bin Arzt!} antworten?\\
Hm, ich auch, aber da hilft dir dieses Nebenfach nicht wirklich weiter. Auch wenn der Name es suggestiert, \fach{Nuklearmedizin} macht dich nicht zum Mediziner. Du lernst zwar etwas \"uber die Grundlagen der menschlichen Anatomie (Leber, Schilddr\"use und Co), aber im Wesentlichen geht es um die physikalischen Eigenschaften von Radionukliden und ihre Anwendungsm\"oglichkeiten in der Medizin. Hierzu besch\"aftigst du dich z.B. mit folgenden Fragen:
\begin{itemize}
  \item{Warum strahlen manche Stoffe?}
  \item{Was ist Strahlung \"uberhaupt und welche Varianten gibt es?}
  \item{Welche Strahlung wirkt wie auf die Zellen und warum?}
  \item{Wie schaffe ich es, dass genau der Bereich verstrahlt wird, den ich bestrahlen will und sonst keiner?}
  \item{Wie nutze ich Strahlung zur Diagnostik? (Detektorphysik)}
  \item{Wie lese ich CT/MRT oder PET Bilder?}
\end{itemize}
Und vielen weiteren. Alles in allem stellt es eine gelungene Symbiose aus Medizin, Zellbiologie und Physik dar und hilft dir, \"uber den (sowieso schon gro\ss en, aber immer noch erweiterbaren) Physikerhorizont hinaus zu blicken.\\\\
Jetzt zu den Fakten:\\
Das Nebenfach muss \"uber 2 Semester belegt werden und umfasst pro Semester 4 SWS, die anwesenheitspflichtig sind. Dazu kommt in der vorlesungsfreien Zeit nach dem 2. Semester ein zweiw\"ochiges Praktikum in der Uniklinik oder im Klinikum Hanau, in dem du das Gelernte anwenden und vertiefen kannst.\\
Abgeschlossen wird \fach{Nuklearmedizin} mit einer m\"undlichen Pr\"ufung,
in der jeweils 15 min der medizinische und physikalische Bereich abgefragt werden. Von den 20CP können bis zu 6 CP als Wahlpflicht eingebracht werden.
\von{Jonas}