\section{Hochschulsport}
Ihr haltet euch für sportlich und auch relativ fit?\\
Tja, liebe Physikeinsteiger, verabschiedet euch von dieser Überzeugung, denn schon bald werdet ihr euren Hintern höchstens vom Bus in den Hörsaal, vom Hörsaal zum Tutorium und von dort ins Praktikum wuchten, wo ihr mit krummen Rücken über dem Schreiber der Triode kauert.\\
Wenn ihr dann nach einem 8-Stundentag nach Hause kommt, und eigentlich die doofen Matheübungen lösen solltet, habt ihr bestimmt keine Lust auf irgendeine Art von geistiger und körperlicher Betätigung. So nimmt der körperliche Zerfall stetig zu.\\
Gerade deshalb solltet ihr euch bei mindestens einem Kurs des Hochschulsports anmelden. Dort könnt ihr aus einem breiten Spektrum, von Afro--Dance bis Zen--Meditation, einen Kurs wählen, der euch geeignet erscheint.
Natürlich gibt es auch allgemeinpopuläre Sportarten. Und das Beste ist: Ihr seid ab nur fünf Euro pro Kurs und Semester dabei.\\
Dem besagten krummen Rücken könnt ihr zum Beispiel mit dem Kurs \enquote{Rückenfit} entgegenwirken. Ihr meint, das ist nur was für Leute, die schon damals in der Schule wegen ihrer Wirbelsäulenfehlstellung gehänselt wurden? Falsch gedacht würde ich meinen, denn nach kurzer Zeit ist die junge, wirklich hübsche Kursleiterin (die wahrscheinlich der Grund für die hohe Männerquote ist) von der netten schüchternen Dame zum Fitnesscoach mutiert, welcher euch mit hunderten Varianten von Sit-Ups traktiert, sodass ihr nicht wisst, ob der Schmerz im Bauch oder etwa der Druck in eurer Birne stärker ist.\\
Gegen den ständigen Input an neuen Informationen und die Reizüberflutung (z.B. in Mathe Satz von: Tonelli, Schwarz, Rolle,\ldots ) ist es auch mal nötig, sich einfach mal auf das \enquote{Nichts} zu konzentrieren. Manche der Studienkollegen haben sich ihre natürlichen Schutzmechanismen wahrscheinlich schon frühzeitig während der Vorlesung selber angeeignet. Jedoch ist für diejenigen, die noch immer die Stimme des Profs wahrnehmen, einer der Yoga-Kurse geeignet. Hier lernt ihr nicht nur euren Kopf zu leeren, sondern auch, euch zu entspannen und euch beweglich zu halten.
An der \enquote{Sportuni} gibt es z.B. normales Hatha--Yoga und Power--Yoga.\\
Ich habe bisher nur von den mehr gesundheitsorientierten Sportarten berichtet, jedoch könnt ihr euch auch gerne in der Fitnesshalle allen guten Ratschlägen der Sportmedizinern widersetzen und einfach mal \enquote{euren Body shapen} (mehr oder weniger), denn wer braucht schon heile Gelenke wenn man Muckis hat? Es kann dann natürlich passieren, dass kompetentes Fachpersonal eingreift und euch daran hindert noch mehr Gewichte aufzulegen. Ein nettes Angebot für den weiblichen Teil der Schöpfung ist das \enquote{Angeleitete Gerätetraining für Frauen}, welches wenig Stress verspricht.\\
Weitere Sportarten sind natürlich diverse Ballspiele (Rugby verspricht wahrscheinlich auch eine gute Anzahl von blauen Flecken), eine Reihe von Kampfsportarten, verschiedene Tanzarten, Leichtathletik usw\ldots\\
Ich bin eventuell nicht sonderlich repräsentativ, da ich nur seltsame Sachen wie Power-Yoga, Ballett, Allwetterlauf oder Rückenfit gemacht habe. Aber ich habe auch von anderen fast nur Positives über den Hochschulsport gehört. Die Lehrkräfte sind wirklich kompetent, verstehen was von ihrem Fach und sind meistens auch sehr nett. Die Preise sind wie gesagt kaum zu unterbieten.\\
Ich kann deshalb jedem empfehlen, sich für mindestens einen Kurs anzumelden. Ihr könnt die Kurse die ersten beiden Wochen ohne Anmeldung besuchen und so testen, was euch Spa\ss  macht.\\
Das Programm erscheint in der Regel ca. zwei Wochen vor Vorlesungsbeginn, dies ist auch als E-Paper im Internet vorhanden. Kleiner Rat am Rande: Geht nicht am ersten Anmeldungstag los, um euch anzumelden. Denn in der Regel ist ein riesiger Haufen von Studenten da, die auch nicht davor scheuen, Gewalt anzuwenden um möglichst schnell einen Platz in einem der sehr beliebten Kurse zu bekommen. Wollt ihr aber einen sehr populären Kurs besuchen, müsst ihr euch wohl oder übel ins Gefecht stürzen. Ich wei\ss  jedoch nicht, welche Kurse davon betroffen sind (Ballett für Fortgeschrittene höchstwahrscheinlich nicht?).\\
Das aktuelle Programm der \enquote{Sportuni} findet ihr auf der Homepage des Zentrums für Hochschulsport unter \url{http://web.uni-frankfurt.de/hochschulsport/}. Dort könnt ihr euch auch zu Beginn der Vorlesungszeit für die einzelnen Sportkurse anmelden.\\
Die meisten Kurse finden übrigens im \textbf{Zentrum für Hochschulsport, Ginnheimer Landstr. 39} statt. Man erreicht es sehr bequem mit dem Bus 34 (\enquote{Universitäts-Sportanlagen}) oder der Stra\ss enbahn 16 (\enquote{Frauenfriedenskirche}). Dort liegt auch das Programmheft mit den allgemeinen organisatorischen Hinweisen aus.\\\\
Also dann, viel Spa\ss !
\von{Nata}
\begin{figure}[!h]
	\centering
  	\includegraphics[height=3in]{\imgdir/thatsitkerl.jpg}
\end{figure}