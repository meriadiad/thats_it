\section{Unistruktur}
%
%
\subsection{Einleitung}
Die Universität verwaltet sich selbst. 
Das bedeutet, sie muss selbst zusehen, wie sie mit den ihr zur Verfügung gestellten finanziellen Mitteln auskommt 
und entscheiden, was sie damit macht. 
Das ist gar nicht so einfach, schließlich haben nicht nur satte 16 Fachbereiche ihre eigenen Interessen, 
sondern auch zahlreiche weitere Einrichtungen, wie etwa die Bibliothek, wollen hier mitreden.

\subsection{Uniweites}
Um das alles unter einen Hut zu bekommen und die Universität nach außen zu vertreten, 
gibt es eine Unileitung, das Präsidium, und einen Chef, den Präsidenten.\\
Weil der nicht alles alleine machen möchte, gibt es noch ganze 4 Vizepräsidenten (darunter auch eine Physikerin), 
einen Kanzler und jede Menge Ausschüsse.
In diesen Gremien sind sämtliche Angehörige der Universität, also Professoren, 
Wissenschaftliche und Administrativ-Technische Mitarbeiter und natürlich auch wir, die Studierenden, vertreten.
Allerdings ist das nur halb so demokratisch, wie es scheint, da die Professoren überall die absolute Mehrheit an Stimmen haben. 
Jedoch sind sie sich auch nur halb so einig, wie man vielleicht denken mag.

Die beiden wichtigsten Gremien sind:

\subsubsection{Der Senat}
Der Senat ist das wichtigste Gremium auf Uni--Ebene.
Er entscheidet über die Entwicklung und die Forschungsschwerpunkte der Uni.
Der Senat besteht aus 9 Professoren, 3 Studierenden, 3 Wissenschaftliche Mitarbeiter und 2 Admininistrativ-Technische Mitarbeiter.

\subsubsection{Der Hochschulrat}
Der Hochschulrat ist eine recht neue Erfindung.
Der Hochschulrat übernimmt seit die Uni Frankfurt eine Stiftungsuni ist die Kontrollfunktion, die damals das Ministerium für Wissenschaft und Kunst des Landes Hessen innehatte.
Dem Hochschulrat gehören elf externe Mitglieder aus den Bereichen Wissenschaft, Wirtschaft und Kunst an.
Er hat das Initiativrecht zu grundsätzlichen Angelegenheiten der Hochschulentwicklung, benennt und entlastet den Präsidenten und bildet aus den eigenen Reihen
(+ einem Vertreter aus dem Ministerium der Finanzen) einen Wirtschafts- und Finanzausschuss.

\newpage
\subsection{Studierendenvertretung}
Darüber hinaus gibt es dann aber noch die studentische Selbstverwaltung, 
das ist das Studierendenparlament (StuPa) und der Allgemeine Studierendenausschuss (AStA).

\subsubsection{Das StuPa}
Hier sitzen ausschließlich Studierende. 
Da diese zur Gruppenbildung neigen, gibt es, ähnlich den Parteien und zum Teil mit diesen verbunden, 
verschiedene Listen, die die Mitglieder des StuPas stellen oder dies gerne würden.\\
Zur Zeit sind dies die Grüne Hochschulgruppe, Juso-Hochschulgruppe, Giraffen - Unabhängige
Fachbereichsgruppen, RCDS - Ring Christlich-Demokratischer Studenten, 
Schildkröten, LHG - Liberale Hochschulgruppe, attac/independent students, 
Die Linke.SDS, Pinguine, LiLi - Wahlbündnis Linke Liste,
DL - Demokratische Linke Liste, FDH - Fachschafteninitiative Demokratische Hochschule und noch einige mehr.\\
Glücklicherweise erhalten nicht alle dieser Gruppen ausreichend Stimmen, um Mitglieder 
ins StuPa entsenden zu können.

\subsubsection{Der AStA}
Der AStA ist so etwas wie die Regierung des StuPas.
Er besteht aus sogenannten Referaten, die sich um die Belange der Studierenden zu unterschiedlichsten Themen kümmern.
Für die Arbeit der Referate stehen dem AStA etwa 500.000 \euro{} pro Jahr zur Verfügung, 
zum Teil auch durch die studentischen Beiträge finanziert.
%
%
\subsection{In der Physik}
Auch der Fachbereich besitzt eine eigene Verwaltung, sein Chef ist der Dekan.

\subsubsection{Der Fachbereichsrat (FBR)}
Äquivalent zum Senat auf Uniebene entscheidet der FBR über die Belange des Fachbereichs Physik.
Den Vorsitz hat der Dekan, der auch den gesamten Fachbereich nach außen vertritt.
Der Fachbereichsrat richtet für alles mögliche Ausschüsse ein, 
beispielsweise zur Erarbeitung der Studien-- und Prüfungsordnung.

\subsubsection{Der Fachschaftsrat (FSR)}
Alle Physikstudierenden gemeinsam sind die Fachschaft.
Diese wählt jedoch Vertreter, die dann den Fachschaftsrat bilden.
Der FSR koordiniert die Arbeit der aktiven Fachschaftler.
Diese beraten und informieren die Studierenden hinsichtlich ihres Studienfaches, 
organisieren z.B. die Einführungsveranstaltung für Erstsemester und drucken dieses Heft.
%
\von{Alex}
