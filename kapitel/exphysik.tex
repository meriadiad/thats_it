\section{Experimentelle Physik}

\begin{tabular}{p{0.2\linewidth}p{0.79\linewidth}}
  \textbf{Was:} & Mechanik und Thermodynamik\\
  \textbf{Wer:} & Prof. Dr. Ren\'e Reifarth, Dr. Kathrin Göbel\\
  \textbf{Wann/Wo:} & Mo: 11:00--13:00, Do: 10:00--12:00, Fr: 13:00-14:00, OSZ H1\\
\end{tabular}\\
\rule{\textwidth}{0.1pt}
\bigskip 

Das Vorlesungsprogramm zur experimentellen Physik wird mit dieser zweigeteilten Vorlesung zur Mechanik (bis Weihnachten) und zur Thermodynamik eröffnet.
In den darauf folgenden Semestern schließen sich dann Vorlesungen zur Elektrodynamik, zur Optik sowie zu den Themen "`Atome und Quanten"', "`Kerne und Elementarteilchen"' und "`Festkörper"' an.
Innerhalb der ersten vier Semester wird so in den Pflichtmodulen \Modul{VEX1A/B} bis \Modul{VEX4A/B} ein Überblick über die grundlegenden Erkenntnisse der Physik gegeben.
Zusätzlich werden Sie schon in den ersten Semestern auch viele Fragestellungen der aktuellen Physik kennenlernen.

Gerade zu Beginn des Studiums, in den ersten Vorlesungen, ist es wichtig, die Modell- und Begriffsbildung in der Physik anhand von Beispielen sorgfältig zu lernen und damit das Rüstzeug für ein tieferes Eindringen in die Physik zu erwerben.  

Zu Anfang der Vorlesung werden Ihr unterschiedlicher Wissensstand und Ihre unterschiedlichen Voraussetzungen aus der Schule noch eine Rolle spielen.
Die Vorlesung wird das berücksichtigen und wesentlichen Begriffe und Konzepte grundlegend behandeln.
Mit zahlreichen Experimenten in der Vorlesung sollen die Zusammenhänge noch besser veranschaulicht werden.
 
Besonders wichtig sind die die Vorlesung begleitenden Übungsaufgaben und Übungsgruppen (Tutorien).
Diese kleineren Übungs- und Lerngruppen finden einmal die Woche statt.
In der Vorlesung werden wöchentlich Übungsaufgaben ausgegeben, die Sie allein oder in Zweierteams bearbeiten sollen.
Die Lösungen der Aufgaben werden dann in den zweistündigen Tutorien besprochen.
So können Sie den Vorlesungsstoff am besten vertiefen. 

Für die Übungen melden Sie sich über das Webportal https://olat.server.uni-frankfurt.de an (siehe unten). Das Webportal ist
ab Donnerstag, 19.10.2017, 20:00 Uhr freigeschaltet.
Sie können zwischen verschiedenen Terminen wählen.
Eine Vorbesprechung zu den Übungen werden wir in der ersten Vorlesungsstunde durchführen.

Nach dem Ende der Vorlesungszeit finden zwei jeweils einstündige Klausuren statt. 
Die Termine werden in der Vorlesung bekannt gegeben.
Die erste Klausur (zur Mechanik) muss nur bestanden werden, sie bleibt unbenotet. 
Die Klausur zur Thermodynamik wird dagegen auch benotet.
Auch zu allen weiteren Vorlesungen des Zyklus finden benotete Modulabschlussprüfungen statt.
Zur Ermittlung der Bachelorgesamtnote am Ende des Bachelorstudiums kann allerdings eine Note aus der Wertung gestrichen werden.

Zu Beginn des Studiums wird nun vieles neu für Sie sein.
Lassen Sie sich von den formalen Regeln und Anforderungen eines Studiums, den Übungsaufgaben, Punkten, Klausuren und Prüfungen Ihren Spaß an der Physik nicht trüben.
Versuchen Sie immer das Studiensystem als angeleitetes Lernen zu sehen, damit Sie in möglichst kurzer Zeit mit den Begriffen und Methoden vertraut werden und das notwendige Werkzeug erwerben.

Nehmen Sie sich bei der Jagd nach den "`Credit-Points"' Zeit, um Ihre eigenen physikalischen Fragen und Ideen zu finden, versuchen Sie die Konzepte und Methoden wirklich zu verstehen und nicht nur für eine Klausur oder Prüfung zu lernen.
So sammeln Sie persönliches Wissen, Kenntnisse und Fertigkeiten. 

Einen guten Start für Ihr Studium.
\smallskip

\noindent
\textbf{Unterlagen zur Vorlesung}

\medskip
Der Stoff der Vorlesung wird an der Tafel entwickelt und mit Folienprojektion unterstützt.
Eine Mitschrift während der Vorlesung wird empfohlen.
Im Verlauf der  wird ein Skript zur Vorlesung angeboten
Eine kurze Einführung in "`Physik Online"'  wird in der Vorlesung gegeben.
Zur Vorlesung gibt es verschiedene Lehrbücher, die didaktisch hervorragend in das Thema einführen.
Parallel zur Vorlesung wird eine begleitende Lektüre sehr empfohlen.

Lehrbuchempfehlungen in alphabetischer Reihenfolge:

\noindent
\textbf{Literatur:}
\begin{itemize}
  \item Bergmann Schäfer, Band 1, Mechanik, Relativität, Wärme, Walter de Gruyter
  \item Demtröder, Experimentalphysik 1, Mechanik und Wärme, Springer Verlag
  \item Dransfeld, Kienle, Kalvius, Physik 1, Mechanik und Wärme, Oldenbourg Verlag
  \item Giancoli, Physik, einbändig, Pearson Studium
  \item Herrmann, Skripten zur Vorlesung Experimentalphysik - Mechanik, online
  \item Meschede, Gerthsen, Physik, einbändig, Springer Verlag
  \item Recknagel, Physik, Band 1, Mechanik, Verlag Technik Berlin
  \item Recknagel, Physik, Band 2, Schwingungen und Wellen, Wärmelehre, Verlag Technik Berlin
  \item Stöcker, Taschenbuch der Physik, Verlag Harri Deutsch   
  \item Tipler, Mosca, Physik für Wissenschaftler und Ingenieure, einbändig, Spektrum Verlag
\end{itemize}

Die Universitätsbibliothek hält einige Titel in großer Anzahl zur Ausleihe zu Verfügung, so dass Sie das/die für Sie passendste/n Buch/Bücher kostengünstig finden und nutzen können.
\noindent
\textbf{Anmeldung zur Übung:}

Sie können sich zu den Übungen auf dem OLAT System des HRZ anmelden.
Nach dem Einloggen in OLAT mit HRZ-Benutzername und Passwort klicken Sie unter "`OLAT-Schnellstart-Links"' auf "`Katalog"' und folgen dem Pfad:\\
$\hookrightarrow$ Lehrveranstaltungen des Fachbereichs 13 - Physik\\
$\hookrightarrow$ Bachelor- /Master-Studiengang "Physik"\\
$\hookrightarrow$ Bachelor-Studium "Physik"\\
$\hookrightarrow$ Pflichtveranstaltungen.\\
$\hookrightarrow$ Dann auf "`Übung zur Experimentalphysik 1"' klicken und für eine Übungsgruppe einschreiben.

\von{Prof. Dr. Ren\'e Reifarth}

