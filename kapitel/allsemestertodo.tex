\section{How to Veranstaltungen und Prüfungen}
\subsection{Den Stundenplan selbst erstellen}
Jetzt haben wir euch zwar schon einen wundervollen Stundenplan zusammengestellt, vielleicht habt ihr aber Lust, noch ein Nebenfach zu belegen, wollt doch lieber mit theoretischer Physik 4 anfangen (das ist keine Empfehlung!) oder fragt euch, wie man sich zu diesen Fächern denn nun anmeldet. So oder so werdet ihr spätestens im nächsten Semester selbst verantwortlich dafür sein, die richtigen Veranstaltungen zusammen zu suchen. Hierzu werdet ihr zwei Instrumente besonders brauchen:\\
Zuallererst und insbesondere benötigt ihr das \textbf{Modulhandbuch} eures Studienganges. Dieses findet ihr, ebenso wie die Prüfungsordnung, auf der Homepage des Prüfungsamtes (\url{https://www.uni-frankfurt.de/60644473/Pruefungsamt}, wobei sich die wenigsten erfahrenen Studenten durch die Uni-Seiten klicken sondern einfach die Suchmaschine ihrer Wahl bemühen). Beide Dokumente solltet ihr in jedem Falle mal gelesen haben, denn hier stehen die wichtigsten Informationen zu eurem Studium. So findet ihr im Modulhandbuch ausführlich eine Beschreibung zu jedem angebotenen Modul (\fach{Fach} das ggfs. aus mehreren verschiedenen Vorlesungen oder Veranstaltungen bestehen kann), dessen Inhalte, benötigte Vorkenntnisse und einer Empfehlung, in welchem Fachsemester es zu belegen ist. Ihr braucht euch an diese Empfehlung nicht zu halten und in vielen Fällen ist dies mit einem Studienbeginn im Sommersemester auch gar nicht möglich, es gibt euch aber einen guten Überblick. Damit ihr wisst, in welcher Reihenfolge die Fächer für euch gedacht sind, erhaltet ihr auf der Erstsemester-Einführung noch einen Studiengangplan für den Studienbeginn im Sommersemester. Mit den nun erlesenen Informationen geht es jetzt weiter auf \url{qis.server.uni-frankfurt.de}, denn dort befindet sich das \textbf{Vorlesungsverzeichnis}.\\
Bei der Abholung eurer Goethe-Card solltet ihr einen Benutzernamen für euren HRZ-Account sowie per Post ein Passwort erhalten haben. Mit diesen könnt ihr euch im QIS einloggen und ab sofort könnt ihr den Großteil eurer Uni-Formalitäten hier regeln. Zur Erstellung eines Stundenplans geht es weiter über \fach{Veranstaltungen} und \fach{Vorlesungsverzeichnis}, hier muss man sich ein bisschen durchklicken, bis man den richtigen Fachbereich (13), den richtigen Studiengang (z.B. Bachelor \fach{Physik}) und den passenden Studienabschnitt (z.B. \fach{gemeinsame Pflichtveranstaltungen}) eingestellt hat. So kommt man letztendlich zur Fachübersicht, wo ihr alle Pflichtvorlesungen, Übungen, Ergänzungskurse etc. zu allen Pflichtveranstaltungen aufgelistet seht, die in diesem Semester angeboten werden. Zum Glück braucht ihr ja nicht alle in einem Semester belegen sondern sucht euch einfach diejenigen zusammen, für die ihr euch mit dem Modulhandbuch entschieden habt. Auf den einzelnen Unterseiten seht ihr nun noch einmal verschiedene Informationen zu den Veranstaltungen, insbesondere an welchem Wochentag und zu welcher Uhrzeit sie angeboten werden. Hierbei gilt es vorsichtig zu unterscheiden zwischen mehreren festen Terminen die Woche (meist bei Vorlesungen), einem festen einmaligen Termin (z.B. Einführungsveranstaltungen) oder vielen Optionen von denen am Ende eine gewählt werden soll (häufig bei Übungen). \\
Zur Übersicht kann man sich die festen Termine vom QIS in einen Stundenplan zusammentragen oder für Outlook exportieren lassen. Für den Export klickt ihr auf der Veranstaltungsseite auf das Kalendersymbol neben dem jeweiligen Termin. Für die Erstellung eines Stundenplans im QIS markiert ihr die Termine und klickt auf \fach{markierte Termine vormerken}. Ihr werdet dann auf einen Stundenplan weitergeleitet, in dem alle in dieser Sitzung vorgemerkten Termine angezeigt werden. Wenn ihr ihn dauerhaft speichern wollt könnt ihr oben auf \fach{Plan speichern} klicken. Es erscheint nun eine Warnung, dass die Veranstaltungen nur vorgemerkt, aber nicht belegt sind.
%
%
\subsection{Belegen von Veranstaltungen}
Im Fachbereich 13 gibt es für \textbf{Vorlesungen} im Allgemeinen keine Belegpflicht. Ihr braucht also nur die passenden Termine im Vorlesungsverzeichnis zusammentragen, damit ihr wisst, wann ihr wo sein müsst (wollt), könnt dann aber einfach zu den Vorlesungen kommen ohne das irgendwo anzumelden. Dies gilt nicht zwingend auch für Nebenfächer oder Fächer, die von anderen Fachbereichen angeboten werden, bei Nebenfächern also immer lieber mal nachfragen. \\
Bei \textbf{Übungen} wird meistens in der ersten Vorlesung erklärt, wie die Anmeldung abläuft, hier gibt es verschiedene Plattformen, verschiedene Zuteilungsmethoden, manche Professoren nutzen hierfür Plattformen wie das elearning-Portal, andere ihre eigene Website oder Zettel an der Tür des Büros. Die Anmeldung ist hier im Allgemeinen Pflicht und oft sollte diese auch möglichst schnell erfolgen, wenn man einen begehrten Termin ergattern möchte. In den höheren Semestern sind sie manchmal sogar schon vor Beginn der Vorlesungszeit freigeschaltet, die Frist endet aber immer erst nach den ersten Vorlesungen, also keine Angst. \\
Etwas mehr aufpassen muss man da bei den für euch ab dem 2. Semester relevanten \textbf{Praktika}, denn deren Anmeldung läuft tatsächlich schon vor Beginn des nächsten Semesters aus. Alle Informationen zur Anmeldung und zu den Anmeldefristen findet ihr auf der Seite des Anfängerpraktikums (\url{https://www.uni-frankfurt.de/44824371/A_Praktikum}).
%
%
\subsection{Prüfungsanmeldung}
Studieren schön und gut, eines Tages ist es soweit und auch das entspannteste Fach endet meist mit einer Prüfung. Da ihr euch zu der Vorlesung ja schon gar nicht angemeldet habt, kann euch niemand dazu zwingen, die Prüfung jetzt mit zu schreiben. Im Fachbereich 13 gibt es jedoch den sogenannten \fach{Freiversuch}, wenn ihr also eine Prüfung im ersten dafür vorgesehenen Semester schreibt, zählt sie bei Nichtbestehen nicht als Fehlversuch (\S 34 Prüfungsordnung 2013). Es ist also doch empfehlenswert, die Prüfung nun auch zu schreiben und dafür muss man sich, im Gegensatz zu den Vorlesungen, anmelden. Damit ihr euch anmelden dürft wird der Prof euch zu Beginn des Semesters darüber informieren, welche Voraussetzungen es zu erfüllen gilt (Teilnahme an Übungen, bestimmte Punkteanzahl in den Übungsblättern und ähnliches). Wenn ihr dann gegen Ende des Semesters alles so weit geschafft habt, sind sich die Profs häufig nicht ganz einig, wie man sich denn zu den Prüfungen anmelden soll. Dies ist in der Physik zentral geregelt und passiert ausschließlich über das QIS unter \fach{Prüfungsverwaltung} (haltet eure Tan-Liste bereit). Die Anmeldung ist in den Grundvorlesungen im Allgemeinen bis 1 Woche vor der Prüfung möglich, wer sich dann doch wieder abmelden möchte kann dies bis einen Tag vor der Prüfung ebenfalls im QIS tun. Sollte es Probleme dabei geben könnt ihr euch mit euren Fragen an die netten Damen vom Prüfungsamt wenden. \\
\par 
So, jetzt solltet ihr das Wichtigste zum Studieren wissen. Wenn ihr noch Fragen habt steht euch die Fachschaft jederzeit gerne zur Verfügung. Viel Erfolg! 
\von{Frederike}