\section{Stundenplan}
\label{sec:stundenplan}
Alles schön und gut, in Wirklichkeit interessiert euch doch vornehmlich eines: Wie sieht euer Studium denn nun aus und was müsst ihr vor Beginn des Semesters tun. Damit legen wir jetzt los:\\
Hier seht ihr ihn, vielleicht zum ersten Mal: Euren Neuen und ganz persönlicher Stundenplan. Im ersten Moment kommt er euch vielleicht noch sehr leer vor, wenn ihr ihn mit eurem Stundenplan aus der Oberstufe vergleicht.
Lasst euch davon nicht täuschen, denn dieser Stundenplan mutiert wenn ihr nicht aufpasst schnell zu einem Zeit- und soziale Kontakte- fressenden schwarzen Loch. Denn statt Hausaufgaben habt ihr neben Vorlesungen und Übungen in den Grundvorlesungen Übungszettel, die es zu bearbeiten und abzugeben gilt um die Klausurzulassung zu erreichen, was schnell in zeitintensive Nachbereitung der Vorlesung und Recherche ausartet und mit gewohnten Hausaufgaben wirklich nichts mehr zu tun hat.\\
Euer Stundenplan besteht zusammengefasst aus Vorlesungen, Tutorien und der unsichtbar bleibenden Zeit für Übungszettel.
Da ihr euer Studium im Sommersemester beginnt werdet ihr in eurem ersten Semester nur Vorlesungen aus dem Bereich der Theoretischen Physik mit mathematischer Ergänzung und der Experimentalphysik sowie ggfs. ein Nebenfach belegen. Die \enquote{echte} Mathematik beginnt ihr dann in eurem zweiten Semester, trotzdem solltet ihr euch zum Verständnis der theoretischen Physik zumindest in der Vorlesung \enquote{Mathematische Ergänzung zur theoretischen Physik} bereits ausführlich damit auseinander setzen.\\
Vorlesungen gehen jeweils über 2 Stunden und es besteht keine Anwesenheitspflicht. Lasst euch davon aber nicht täuschen, denn wenn ihr es ernst mit dem Studium meint kommt ihr um den Besuch der Vorlesung kaum herum. Es ist zwar schwer, dem Menschen an der Tafel zu folgen, noch viel schwerer ist es aber, den Vorlesungsstoff für die Übungsaufgaben und die Klausur komplett eigenständig zu erarbeiten. Vor Allem könnt ihr den Profs in den Vorlesungen Fragen stellen, wohingegen die meisten Bücher einen daraufhin nur hämisch anschweigen oder auf der nächsten Seite darauf verweisen, dass dies \enquote{offensichtlich} aus Satz 193b.IV des Buches \enquote{Quantenchromoelektrodynamik für Fortgeschrittene} folge. Dann doch lieber 2 Stunden Vorlesung.\\
Um das was in der Vorlesung theoretisch in euer Gehirn eingeflößt wurde mal praktisch angewendet zu haben sowie zur Nachbesprechung der Übungszettel habt ihr sogenannte Tutorien. Hier ist die Anwesenheit (bis auf wenige Ausnahmen) Pflicht, ebenso die aktive Beteiligung durch Mitarbeit und Vorrechnen an der Tafel. Oft ist das Tutorium aber auch der Ort, wo man beginnt, wirklich zu verstehen was das alles soll, denn hier steht euch ein*e Student*in aus höheren Semestern, ein*e Masterstudent*in oder ein*e Doktorand*in zur Seite, hat Zeit, wirklich auf Fragen einzugehen und weiß oft noch von sich selbst oder aus anderen Tutorien, welche Knoten es in den Denkmuskeln zu lösen gilt, welche Methoden wirklich hilfreich zum Verständnis sind und können euch so den Weg zur erfolgreichen Klausur und zum Verständnis sehr erleichtern. Tutorien werden in der Woche zu vielen verschiedenen Terminen angeboten, die genauen Termine werden euch die Professoren oder Übungsleiter in der ersten Vorlesung bekannt geben, zusammen mit den notwendigen Informationen zur Anmeldung.\\
Die letzte Veranstaltung, die noch in Eurem Stundenplan steht ist das physikalische Kolloquium, bei dem es nicht nur Kaffee und Kekse gibt, sondern auch interessante Vorträge über die aktuelle Physik. Auch wenn ihr aktuell vielleicht noch nicht so viel versteht, spannend ist es trotzdem zu sehen, was es so alles gibt und ihr werdet staunen, wie schnell euch diese seltsamen Zeichen an der Tafel anfangen, bekannt vor zu kommen. Es geht kaum etwas über die Freude, im zweiten Semester eine Methode auf einmal wie selbstverständlich anzuwenden, die einen im ersten Semester noch zur Verzweiflung gebracht hat.\\
Wenn euch das jetzt nicht zu sehr abgeschreckt hat und ihr immer noch Lust auf euer Physikstudium für Masochisten habt: viel Spaß beim Lesen eures Stundenplans. Die Informationen, wie und ob man sich für Fächer, Klausuren und Co anmelden müsst erhaltet ihr von uns bei der Erstsemester-Einführung.

\noindent%%%%%%%%%%%%%%%%%%%%%%%%%%%%Physik%%%%%%%%%%%%%%%%%%%%%%%%%%%%
\

\begin{sideways}
\begin{minipage}{1\textheight}
%\begin{tabular}{|c||c|c|c|c|c|}\hline
\centering \textbf{\Large Stundenplan für Bachelor Physik}\bigskip\\
\begin{tabular}{|p{.05\textheight}||p{.16\textheight}|p{.16\textheight}|p{.16\textheight}|p{.16\textheight}|p{.16\textheight}|}\hline
 Zeit 		& Montag 			& Dienstag 			& Mittwoch 		& Donnerstag 		& Freitag	\\ \hline\hline
{08.00} 	& 					&					&  {Ex 2}		& {Theoretikum} 	&  			\\
&&&&&\\ \cline{1-3} \cline{6-6}
{09.00} 	& 					&					&  {OSZ H1}		& {!!! Beispiel !!!}&			\\
&&&&&\\ \cline{1-6}
{10.00} 	& 	 				&  					& {Theo 1/2}	& 					& {Ex 2}	\\
&&&&&\\ \cline{1-3}\cline{5-5}
{11.00} 	& 	 				& 					& {\_0.111} 	&  {Theo 1/2}		& {OSZ H1}	\\
&&&&&\\ \cline{1-4} \cline{6-6}
{12.00} 	& 					& {Theo 1/2}		& 	 			& {\_.102}			& 			\\
&&&&&\\ \cline{1-2} \cline{4-6}
{13.00} 	&					& 	{\_.102}		& 				& 					&  			\\
&&&&&\\ \hline
{14.00} 	& {Mathe Ergänzung} &{Übung Ex 2}		& 				&  					& 			\\
&&&&&\\ \cline{1-1}  \cline{4-6}
{15.00} 	& {OSZ H1}	 		&{!!! Beispiel !!!}	& 				&	  				& 			\\
&&&&&\\ \hline
 {16.00} 	&  					&  					& {Kolloquium} 	&  					& 			\\
&&&&&\\ \cline{1-3} \cline{5-6}
 {17.00} 	&  					&  					& {\_0.111} 	&  					& 			\\
&&&&&\\ \hline
 {18.00} 	&  					&	  				& 		 		&		  			& 			\\
&&&&&\\ \hline
 {19.00} 	&  					&  					&  				& 					& 			\\
&&&&&\\ \hline
\end{tabular}
\end{minipage}
%}
\end{sideways}
%
%
%\noindent%%%%%%%%%%%%%%%%%%%%%%%%%Biophysik%%%%%%%%%%%%%%%%%%%%%%%%%%%%
%\
%
%\begin{sideways}
%\begin{minipage}{1\textheight}
%\center \textbf{\Large Stundenplan für Bachelor Biophysik}\bigskip\\
%\begin{tabular}{|p{.05\textheight}||p{.16\textheight}|p{.16\textheight}|p{.16\textheight}|p{.16\textheight}|p{.16\textheight}|}\hline
% Zeit 		& Montag 	& Dienstag 	& Mittwoch 	& Donnerstag 	& Freitag \\ \hline\hline
%{08.00} 	&  		&  		& {Ex 2}  	&  		&		\\
%&&&&&\\ \cline{1-3} \cline{5-6}
%{09.00} 	&  		&  		& {OSZ H1}	& {Theo 1} 	&		\\
%&&&&&\\ \cline{1-4} \cline{6-6}
%{10.00} 	& 	 	& {Theo 1/2} 	& {Ex -- Übung}	& {\_0.111} 	& 		\\
%&&&&&\\ \cline{1-2} \cline{5-6}
%{11.00} 	& 	 	& {\_0.111}	& {!!! Beispiel !!!} & {Ex 2} 	& 		\\
%&&&&&\\ \cline{1-4} \cline{6-6}
%	{12.00} 	& {Theoretikum}	&  		& {Theo 1/2} 	& {OSZ H1} 	& {Mathematik für Biophysiker} \\
%&&&&&\\ \cline{1-1} \cline{3-3} \cline{5-6}
%	{13.00} 	& {!!! Beispiel !!!} & 		& {\_0.111} 	& {Mathe-Übung !!!Beispiel!!} & {\_\_.401} \\
%&&&&&\\ \hline
%{14.00} 	& {Mathe Ergänzung} & 		& 		&  		& 		\\
%&&&&&\\ \cline{1-1} \cline{3-6}
%{15.00} 	& {\_0.111}	 &  		& 		&	  	& 		\\
%&&&&&\\ \hline
% {16.00} 	&  		&  		& {Kolloquium} 	&  		& 		\\
%&&&&&\\ \cline{1-3} \cline{5-6}
% {17.00} 	&  		&  		& {\_0.111} 	&  		& 		\\
%&&&&&\\ \hline
% {18.00} 	&  		&  		&  		&  		& 		\\
%&&&&&\\ \hline
% {19.00} 	&  		&  		&  		&  		& 		\\
%&&&&&\\ \hline
%\end{tabular}
%\end{minipage}
%%}
%\end{sideways}
