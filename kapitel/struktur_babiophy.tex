\subsection{Bachelor in Biophysik}
\bigskip
%
Herzlich willkommen zum Biophysik-Studium an der Uni Frankfurt.
Mit eurer Einschreibung hier habt ihr euch für einen spannenden interdisziplinären Studiengang an der Schnittstelle zwischen \fach{Physik}, \fach{Chemie} und \fach{Biologie} entschieden.\\
"`Interdisziplinär"' heißt, dass euer Studiengang offiziell von drei Fachbereichen zusammen ausgerichtet wird dem FB 13 (Physik) als dem federführenden Fachbereich, hier seid ihr sozusagen zu Hause, dem FB 14 (Biochemie, Chemie und Pharmazie) und dem FB 15 (Biowissenschaften), an denen ihr auch sehr viele Veranstaltungen besuchen werdet.\\
\bigskip\\
%
Das Ziel der Biophysik ist es, biologische Systeme und Prozesse mit Methoden der Physik zu untersuchen und zu beschreiben.
Es handelt sich also keinesfalls um eine Physik light!\\
Nachdem ihr die Grundlagen der drei Naturwissenschaften gemeistert habt,
geht es an die spannenden Fragestellungen der modernen Biophysik, hier in Frankfurt geht es dabei besonders um molekulare Systeme. 


\subsubsection{Erstes Semester}

Im ersten Semester habt ihr vor allem Grundlagenvorlesungen und eine kleine Einführung in die Themen der Biophysik. \\
\bigskip\\
%
In der \VL{Theoretische Physik 1 -- Mathematische Methoden} lernt ihr zusammen mit den Physikern die nötigen Grundlagen,
um im zweiten Semester richtig mit der theoretischen Physik loslegen zu können.
Ebenfalls zusammen mit den Physikern hört ihr die \VL{Experimentalphysik 1a -- Mechanik}.
Diese Vorlesung ist ebenso wie die Theorievorlesung unbenotet,geht jedoch nur bis Weihnachten.
Anders als die Physikstudenten hört ihr nicht direkt im Anschluss die Thermodynamikvorlesung.\\
\bigskip\\
%
In der \fach{Chemie} hört ihr mit anderen Nebenfächlern die \VL{Allgemeine und Anorganische Chemie}.
Am Ende gibt es eine benotete Klausur.
Eure Grundlagenvorlesung bei den Biologen ist die \VL{Struktur und Funktion der Organismen},
auch hier müsst ihr eine unbenotete Klausur schreiben.\\ 
\bigskip\\
%
Damit ihr einen Eindruck bekommt, für was das alles mal gut sein wird, gibt es die \VL{Einführung in die Biophysik}.
Diese einstündige Vorlesung ist unbenotet.

\subsubsection{Zweites Semester}

Im zweiten Semester beginnt ihr in der \fach{Theoretischen Physik} mit der \VL{Mechanik}.
Zusätzlich besucht ihr die \VL{Mathematik für Biophysiker}, eine kleine unbenotete Mathevorlesung mit 3 CP.
In der \fach{Experimentalphysik} wird nun die \VL{Elektrodynamik} behandelt.
Beide Vorlesungen hört ihr zusammen mit den Physikern.
Sie bringen euch jeweils 8CP und sind benotet.\\ 
\bigskip\\
%
Außerdem geht es jetzt in der \fach{Chemie} richtig los.
Ihr hört die organische und die physikalische Chemievorlesung gemeinsam mit den Studenten der Chemie.
Die unbenotete \VL{OC} bringt euch 7CP und die benotete \VL{PC} 6CP.
In der organischen Chemie geht es um Strukturen und Reaktionen. Das Thema der physikalischen Chemie in diesem Semester ist die Thermodynamik.

\subsubsection{Drittes Semester}

Im dritten Semester kommt dann das erste Praktikum, und zwar das \VL{Anfängerpraktikum} der Physik.
Für euch dauert es anders als für die Physiker nur ein Semester.
Trotzdem behandelt ihr alle Themengebiete.
Da ihr zu jedem der beiden Teile nur die Hälfte der Physikerversuche macht, wird nach der Hälfte des Semesters gewechselt.\\ 
\bigskip\\
%
Ihr habt nun eure letzten Experimentalphysikvorlesungen \VL{Atome und Quanten} und \VL{Optik}.
Beide haben nur 4CP und sind benotet.
In der \fach{Theoretischen Physik} kommt jetzt die \VL{Elektrodynamik} dran.
Und in der \fach{Biologie} hört ihr die Biochemievorlesung.
Diese geht nur bis nur Hälfte des Semesters und bringt euch 3 benotete CP.\\ 
\bigskip\\
%
Außerdem geht es nun mit der eigentlichen Biophysik los.
Ihr habt die \VL{Biophysik 1} Vorlesung.
Hier geht es vor allem um die physikalischen Eigenschaften von biologischen Molekülen, Biopolymeren und Membranen.
Insgesamt habt ihr drei Biophysikvorlesungen jeweils mit 6 CP und einer benoteten Klausur.

\subsubsection{Viertes Semester}

Im vierten Semester findet eure letzte reguläre Theorievorlesung, die \VL{Quantenmechanik}, statt.
Außerdem habt ihr die \VL{Biophysik-2}-Vorlesung zu experimentellen Methoden der Strukturaufklärung. 
Zusätzlich besucht ihr das \VL{Seminar A} zu aktuellen Themen der Biophysik.\\
\bigskip\\
%
In der \fach{Chemie} hört ihr die \VL{Biophysikalische Chemie 2}, welche Enzymkinetik behandelt und zusammen mit den Biochemikern gehört wird.
Als Vorbereitung für das \VL{organisch-chemische Praktikum}, welches ihr in der vorlesungsfreien Zeit nach diesem Semester als Blockpraktikum haben werdet, besucht ihr ein Seminar zur organischen Chemie.
Im Praktikum führt ihr in Zweier- oder Dreiergruppen 8 Versuche durch. 
Dieses Modul wird mit einem benoteten Abschlusskolloquium beendet und setzt die bestandene OC-Klausur voraus.\\
\bigskip\\
%
In der ersten Hälfte dieses Semesters könnt ihr bei den Biologen die \VL{Genetik} Vorlesung hören.

\subsubsection{Fünftes Semester}

In der \fach{Biophysik} fangt ihr mit der \VL{Biophysik 3} an,
hier lernt ihr nun die verschiedenen Methoden zur Analyse von Funktion und Dynamik verschiedener Biomoleküle.
Damit ihr das neue Wissen auch gleich anwenden könnt, habt ihr dieses Semester auch ein weiteres Praktikum und zwar das \VL{Biophysik-Praktikum},
dies entspricht etwa dem Fortgeschrittenen-Praktikum, das eure Physik-Kollegen besuchen.\\
\bigskip\\
%
Ihr habt in diesem Semester noch ein weiteres Praktikum und zwar in der \fach{physikalischen Chemie}.
Hier macht ihr in Zweier-Teams Versuche zu den beiden physikalischen Chemievorlesungen und am Ende steht wieder ein Abschlusskolloquium.\\ 
\bigskip\\
%
Außerdem könnt ihr in diesem Semester eure Vorlesung aus der Biologiespezialisierung hören.

\subsubsection{Sechstes Semester}

Das sechste Semester sollte für die Bachelor-Arbeit zur Verfügung stehen. \\
\bigskip\\
%
Außerdem habt ihr die \VL{Theoretische Chemie 2} Vorlesung.
In der \fach{Biologie} hört ihr noch die \VL{Zellbiologie} Vorlesung
und, falls ihr sie nicht im fünften Semester gehört habt, ist jetzt noch die Biologiespezialisierungsvorlesung dran.

\subsubsection{Prüfungsmodalitäten und Endnote}
Die Bestimmungen für die Anzahl von Prüfungswiederholungen und die Fristen zur Prüfungsanmeldung, sind für die Vorlesungen, der unterschiedlichen Fachbereiche unterschiedlich. 
Die Studienordnung der Biophysik verweist dafür auf die entsprechenden anderen Ordnungen. \\
\bigskip\\
%
Die Endnote setzt sich aus den erhaltenen Noten gewichtet mit der CP Anzahl zusammen.
 

\input{struktur_biophy_tabellews13.tex}
