\section{Anerkannte Beweismethoden der Lehrkörper}


Extra für euch haben wir nun hier die wichtigsten Beweismethoden
zusammengestellt:
\vskip 1 cm

\begin{description}
    \item[METHODE DER EXAKTEN BEZEICHNUNGEN:] Sei P ein Punkt Q,
    den wir R nennen.

    \item[AUTORITÄTSGLÄUBIGE METHODE:] Das
    muss stimmen. Das steht so im Bronstein.
    
    \item[KAPITALISTISCHE METHODE:] Eine Gewinnmaximierung tritt ein, wenn
    wir gar nichts beweisen, dann verbrauchen wir nämlich am wenigsten
    Kreide.
    
    \item[KOMMUNISTISCHE METHODE:] Das beweisen wir jetzt gemeinsam. Jeder
    schreibt eine Zeile, und das Ergebnis ist Staatseigentum.
    
    \item [NUMERISCHE METHODE:] Grob gerundet stimmt's!
    
    \item [BEWEIS DURCH EINSCHÜCHTERUNG:] Das ist doch wohl trivial!
    
    \item [BEWEIS DURCH ÜBERLADENE NOTATION:] Am besten verwendet man mindestens
    vier Alphabete und viele Sonderzeichen. Hier reicht das griechische
    Alphabet alleine nicht mehr aus, um engagierte Zuhörer
    abzuschrecken. Ein kurzer Exkurs in die hebräischen Sonderzeichen
    sollte aber auch den stärksten Zweifler zum Schweigen bringen.
    
    \item [BEWEISE DURCH AUSLASSEN:] 1. Die Details bleiben als leichte
    Übungsaufgabe dem geneigten Leser überlassen. 2. Die anderen 253
    Fälle folgen völlig analog hierzu.
    
    \item [BEWEIS DURCH REDUKTION AUF DAS FALSCHE PROBLEM:] Um zu zeigen, dass
    dies eine Abbildung in die Menge der s-saturierten Ideale ist,
    reduzieren wir es auf die Riemannsche Vermutung.
    
    \item [BEWEIS DURCH REKURSIVEN QUERVERWEIS:] In Quelle a wird Satz 5
    gefolgert aus Satz 3 der Quelle b, welcher seinerseits sofort aus
    Korollar 6.2 der Quelle c folgt, den man trivial aus Satz 5 der
    Quelle a erhält.
    
    \item [BEWEIS DURCH SCHEINVERWEIS:] Nichts dem zitierten Satz auch nur
    entfernt ähnliches erscheint in der angegebenen Quelle.
    
    \item [3-W-METHODE:] Wer will's wissen?
    
    \item [BEWEIS DURCH PAUSE:] Prof kurz vor der Pause: Diesen Satz beweise ich
    Ihnen nach der Pause. Prof nach der Pause: Wie wir vor der Pause
    bewiesen haben...
\end{description}
